\clearpage
\section{S}

\subsection{硫化氢}
\subsubsection{物理性质}
\begin{itemize}
	\item 无色、有刺激性气味(臭鸡蛋味)、有毒气体
	\item 能溶于水
	\item \ce{H2S}水溶液俗称氢硫酸,有毒
	\begin{itemize}
	\item 碘酸:\ce{HIO}
	\item 碘化氢:\ce{HI}
	\item 氢碘酸:\ce{HI}水溶液
	\end{itemize}
\end{itemize}
\subsubsection{化学性质}
\paragraph{弱酸性}
\subparagraph{与碱生成对应酸式/正盐}
\subparagraph{与一些盐反应}
\begin{itemize}
	\item $\ce{H2S + CuSO4 -> CuS v + H2SO4}$(强酸置弱酸)
	\item $\ce{PbAc2 + H2S -> PbS v + 2HAc}$(鉴别硫化氢:$\underset{\text{醋酸铅}}{\ce{PbAc2}}$试纸变黑)
\end{itemize}
\paragraph{不稳定性}
高温易分解
\paragraph{可燃性}
\begin{itemize}
	\item $\ce{2H2S + 3O2 ->[\text{点燃}] 2SO2 + 2H2O}$
	\item $\ce{2H2S + O2 ->[\text{点燃}] 2S + 2H2O}$
\end{itemize}
\paragraph{强还原性}
\begin{itemize}
	\item $\ce{2H2S + SO2 -> 3S v + 2H2O}$
	\item $\ce{2H2S(aq) + O2 -> 2S v + 2H2O}$
	\item $\ce{H2S + X2 -> 2HX + S v}$
	\item $\left\{\begin{array}{lr}
			\ce{H2S + H2O2 -> 2H2O + S v}\\
			\ce{H2S + 4H2O2 -> H2SO4 + 4H2O}\\
		\end{array}\right.$
\end{itemize}
\subsubsection{制备}
向上双管排气法收集
除杂:\ce{NaOH}
\begin{itemize}
	\item $\ce{FeS + H2SO4 -> H2S ^ + FeSO4}$
	\item $\ce{ZnS + H2SO4 -> H2S ^ + ZnSO4}$
\end{itemize}

\subsection{硫单质}
\subsubsection{物理性质}
\begin{itemize}
	\item \textcolor[rgb]{0.905,0.803,0.376}{黄色}硫固体/\textcolor[rgb]{0.874,0.890,0.756}{淡黄色}硫粉/白色纳米尺度的硫
	\item 难溶于水、微溶于酒精、易溶于二硫化碳
	\item 熔沸点低,存在多种同素异形体
\end{itemize}
\subsubsection{化学性质}
\paragraph{与金属反应}
主要生成低价化合物
\begin{itemize}
	\item $\ce{S + Fe ->[\Delta] FeS}$
	\item $\ce{S + 2Cu ->[\Delta] \underset{\text{硫化亚铜}}{Cu2S}}$
	\item $\underbrace{\ce{3S + 2Al ->[\Delta] Al2S3}}_{\text{	高中唯一制\ce{Al2S3}的方法}}$
	\item $\underbrace{\ce{S + Hg -> HgS}}_{\text{除汞}}$
\end{itemize}
\paragraph{与非金属反应}
\begin{itemize}
	\item $\ce{S + 3F2 -> \underset{\text{变压器涂层}}{SF6}}$
	\item $\ce{S + O2 ->[\Delta\text{或点燃}] SO2}$
	\item $\ce{S + H2 <=>[\text{高温}] H2S}$
\end{itemize}
\paragraph{还原性}
\begin{itemize}
	\item $\left\{\begin{array}{lr}
			\ce{S + 4HNO3(\text{浓}) ->[\Delta] SO2 ^ + 4NO2 ^ + 2H2O}\\
			\ce{S + 2H2SO4(\text{浓}) ->[\Delta] 3SO2 ^ + 2H2O}\\
		\end{array}\right.$
	\item $\ce{S + 3H2O2 -> H2SO4 + 2H2O}$
\end{itemize}
\paragraph{除硫粉}
\begin{enumerate}
	\item 用\ce{CS2}洗涤
	\item 用热的氢氧化钠溶液洗涤:$\ce{3S + 6NaOH ->[\Delta] 2Na2S + Na2So3 + 3H2O}$
\end{enumerate}
\paragraph{歧化和归中}
\begin{itemize}
	\item 硫单质:$\left\{\begin{array}{lr}
			\underbrace{\ce{S + OH- ->[\Delta] S^2- + SO3^2- + H2O}}_{\text{碱性歧化}}\\
			\underbrace{\ce{S^2- + SO3^2- + H+ -> S v + H2O}}_{\text{酸性归中}}\\
		\end{array}\right.$
	\item 卤素加热: $\left\{\begin{array}{lr}
			\ce{X2 + OH- ->[\Delta] X- + XO3- + H2O}\\
			\ce{X- + XO3- + H+ -> X2 ^ + H2O}\\
		\end{array}\right.$
	\item 卤素不加热:$\left\{\begin{array}{lr}
			\ce{X2 + OH- -> X- + XO- + H2O}\\
			\ce{X- + XO- + H+ -> X2 ^ + H2O}\\
		\end{array}\right.$
\end{itemize}
\subsection{二氧化硫}
\subsubsection{物理性质}
\begin{itemize}
	\item 无色、刺激性气味、有毒气体
	\item 易溶于水
\end{itemize}
\subsubsection{化学性质}
\paragraph{酸性}
\begin{itemize}
	\item 与强碱:$\left\{\begin{array}{lr}
					\ce{SO2 + 2NaOH -> Na2SO3 + H2O}\\
					\ce{SO2 + NaOH -> NaHSO3}\\
					\ce{SO2 + Ca(OH)2 -> CaSO3 v + H2O}\\
				\end{array}\right.$
	\item 与碱性氧化物:$\left\{\begin{array}{lr}
					\underbrace{\ce{SO2 + CaO -> CaSO3}}_{\text{蜂窝煤脱硫}}\\
					\ce{2CaSO3 + O2 ->[\Delta] 2CaSO4}\\
				\end{array}\right.$
	\item 与水反应:$\left\{\begin{array}{lr}
					\ce{SO2 + H2O <=> H2SO3}\\
					\ce{H2SO3 <=> H+ + HSO3-}\\
					\ce{HSO3- <=> H+ + SO3^2-}\\
				\end{array}\right.$
	\item 酸性比盐酸弱:不与\ce{BaCl2}溶液反应生成沉淀
	\item 与\ce{BaCl2}和\ce{NH3*H2O}溶液:$\left\{\begin{array}{lr}
			\ce{SO2 + 2NH3*H2O -> (NH4)2SO3 + H2O}\\
			\ce{(NH4)2SO3 + BaCl2 -> BaSO3 v + 2NH4Cl}\\
		\end{array}\right.$
	\item 与\ce{BaCl2}和\ce{Cl2}溶液:$\left\{\begin{array}{lr}
					\ce{SO2 + 2H2O + Cl2 -> H2SO4 + 2HCl}\\
					\ce{H2SO4 + BaCl2 -> 2HCl + BaSO4 v}\\
				\end{array}\right.$
\end{itemize}
\paragraph{氧化性}
$\ce{2H2S + SO2 -> 3S v + H2O}$(仅此一个反应能体现氧化性)
\begin{itemize}
	\item \ce{SO2}通入\ce{Na2S}溶液:$\left\{\begin{array}{lr}
			\ce{SO2 + H2O -> H2SO3}\\
			\ce{H2SO3 + Na2S -> Na2SO3 + H2S ^}\\
			\ce{2H2S + SO2 -> 3S v + H2O}\\
		\end{array}\right.$
\end{itemize}
\paragraph{还原性}
\begin{itemize}
	\item $\ce{SO2 + H2O2 -> H2SO4}$
	\item $\ce{SO2 + Na2O2 -> Na2SO4}$
	\item $\ce{SO2 + 2Fe^3+ + 2H2O -> 2Fe^2+ + SO4^2- + 4H+}$
	\item $\ce{5SO2 + 2MnO4- + 2H2O -> 2Mn^2+ + 5SO4^2- + 4H+}$
	\item $\ce{SO2 + HClO + H2O -> 3H+ + Cl- + SO4^2-}$
	\item $\ce{NO2 + SO2 -> NO + SO3}$
	\item $\left\{\begin{array}{lr}
			\ce{SO2 + 2H2O + X2 -> H2SO4 + 2HX}\\
			\ce{SO2 + 2H2O + Cl2 -> H2SO4 + 2HCl}\\
		\end{array}\right.$
\end{itemize}
\paragraph{漂白性}
\ce{SO2}使\textcolor[rgb]{0.721,0.207,0.105}{品红溶液}褪色,加热后红色复现。原理:与特定有机染料结合,生成无色或浅色物质;加热可逆
\begin{itemize}
	\item \ce{SO2}通入酸性高锰酸钾溶液褪色:还原性
	\item \ce{SO2}通入品红溶液褪色:漂白性
\end{itemize}
\subsubsection{硫酸型酸雨}
\begin{itemize}
	\item $\ce{SO2 -> SO3 -> H2SO4}$
	\item $\ce{SO2 -> H2SO3 -> H2SO4}$
\end{itemize}
\subsubsection{除杂}
\begin{itemize}
	\item \ce{SO2}(\ce{CO2}):\ce{NaHSO3}溶液
	\item \ce{CO2}(\ce{SO2}):\ce{NaHCO3}溶液或酸性高锰酸钾溶液
	\item \ce{SO2}(\ce{HCl}):\ce{NaHSO3}溶液
	\item \ce{SO2}(\ce{Cl2}):无法分开
\end{itemize}
\subsubsection{制备}
\begin{itemize}
	\item $\left\{\begin{array}{lr}
			\ce{Na2SO3 + H2SO4({浓}) -> Na2SO4 + CO2 ^ + H2O}\\
			\ce{Na2SO3 + H2SO4 ->[\Delta] Na2SO4 + CO2 ^ + H2O}\\
		\end{array}\right.$
	\item 装置:固液加热,含沸石
	\item 除杂(水):浓硫酸或无水氯化钙
	\item 收集:向上排空气(易溶于水,不能用排水法)
	\item 验满:湿润的蓝色石蕊试纸(酸性)或品红试纸(漂白性)
	\item 尾气处理:氢氧化钠溶液、放倒吸(工业用氨水,产物可做化肥)
\end{itemize}

\subsection{三氧化硫}
\subsubsection{物理性质}
\begin{itemize}
	\item 无色
	\item 常温液体、标况固体
	\item 溶于浓硫酸
\end{itemize}
\subsubsection{化学性质}
酸性氧化物,与水反应生成硫酸,放热。
$$
\ce{SO3 + CaO -> CaSO4}\\
\ce{SO3 + 2NaOH -> Na2SO4 + H2O}
$$
\subsubsection{除杂}
弱酸气体混有强酸气体杂质时,用弱酸的酸式盐溶液除杂。也可以利用杂质的氧化性或还原性除杂。
\begin{itemize}
	\item \ce{CO2}(\ce{SO2}):酸性高锰酸钾溶液、\ce{Fe2(SO4)3}溶液、\ce{NaHCO3}溶液
	\item \ce{H2S}(\ce{HCl}):饱和\ce{NaHS}溶液
	\item \ce{CO2}(\ce{H2S}):酸性高锰酸钾溶液、\ce{Fe2(SO4)3}溶液、\ce{CuSO4}溶液
\end{itemize}

\subsection{亚硫酸}
\subsubsection{化学性质}
\paragraph{不稳定性}
\begin{itemize}
	\item $\ce{H2SO3 ->[\Delta] H2O + SO2 ^}$
\end{itemize}
\paragraph{还原性}
\begin{itemize}
	\item $\ce{2H2SO3 + O2 <=> H2SO4}$
	\item $\ce{H2SO3 + Cl2 + H2O <=> H2SO4 + 2HCl}$
	\item $\ce{H2SO3 + H2O2 <=> H2SO4 + H2O}$
\end{itemize}
\paragraph{酸性}
亚硫酸是中强酸
\begin{itemize}
	\item \ce{NaHCO3}:显碱性
	\item \ce{NaHSO3}:显酸性
\end{itemize}

\subsection{硫酸}
\subsubsection{物理性质}
\begin{itemize}
	\item 无色粘稠状液体、不易挥发
	\item 吸水性
	\item 溶于水放热
\end{itemize}
\subsubsection{化学性质}
\paragraph{酸性}
\paragraph{脱水性(注意区分吸水性)}
酸性干燥剂
\paragraph{强氧化性}
\begin{itemize}
	\item 与金属反应:可与金属活动顺序表中铜及之前的物质反应,常温下使铁、铝钝化。
	\begin{itemize}
		\item $\ce{Cu + 2H2SO4(\text{浓}) ->[\Delta] CuSO4 + SO2 ^ + 2H2O}$
	\end{itemize}
	\item 与非金属反应:$\left\{\begin{array}{lr}
			\ce{C + 2H2SO4(\text{浓}) ->[\Delta] CuSO4 + SO2 ^ + 2H2O}\\
			\ce{S + 2H2SO4(\text{浓}) ->[\Delta] 3SO2 ^ + 2H2O}\\
		\end{array}\right.$
	\item 与化合物反应:$\left\{\begin{array}{lr}
			\ce{2Br- + SO4^2- + 4H+ -> Br2 + SO2 ^ + 2H2O}\\
			\ce{2Fe^2+ + SO4^2- + 4H+ -> 2Fe^3+ + SO2 ^ + H2O}\\
		\end{array}\right.$
\end{itemize}
\subsubsection{制备}
\paragraph{工业}
\subparagraph{沸腾炉}
煅烧黄铁矿
$$\ce{4FeS2 + 11O2 ->[\Delta] 2Fe2O3 + 8SO2}$$
\subparagraph{接触室}
\ce{V2O5}附着于网上
$$\ce{2SO2 + O2 ->[{催化剂}][\Delta] 2SO3}$$
\subparagraph{吸收塔}
$$\ce{SO3 + H2O -> H2SO4}$$
实际用浓硫酸吸收
$\left\{\begin{array}{lr}
	\ce{H2SO4 + SO3 -> \underset{\text{焦硫酸}}{\ce{H2S2O7}}}\\
	\ce{H2S2O7 + H2O -> 2H2SO4}\\
\end{array}\right.$
\subsection{含硫酸盐}
\subsubsection{\ce{FeSO4}}
$$
\ce{FeSO4 ->[\Delta] Fe2O3 + SO2 ^ + SO3 ^}
$$
\subsubsection{\ce{CuSO4}}
$$\left\{\begin{array}{lr}
	\ce{CuSO4 ->[\Delta] CuO + SO2}\\
	\ce{CuSO4 ->[\Delta\text{(更高温度)}] CuO + SO2 ^ + SO3 ^ + O2 ^}\\
\end{array}\right.$$
\paragraph{制备}
$\left\{\begin{array}{lr}
	\ce{2Cu + 2H2SO4(\text{稀}) + O2 ->[\Delta] 2CuSO4 + 2H2O}\\
	\ce{Cu + H2SO4(\text{稀}) + H2O2 -> CuSO4 + 2H2O}\\
\end{array}\right.$
\subsubsection{\ce{Na2S2O3}}
\begin{itemize}
	\item 无法在酸性条件下存在:$\ce{Na2S2O3 + 2HCl -> 2NaCl + H2O + SO2 ^ + S v}$
	\item 生成:$\ce{Na2So3 + S -> Na2S2O3}$
	\item 除氯剂:$\ce{Na2S2O3 + 4Cl2 + 10NaOH -> 8NaCl + 2Na2So4 + 5H2O}$
	\item 测定空气中\ce{I2}含量:$\ce{2Na2S2O3 + I2 -> B=Na2S4O6 + 2NaI}$
\end{itemize}
