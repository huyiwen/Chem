\clearpage
\section{Si}
\subsection{硅单质}
\subsubsection{物理性质}
\begin{itemize}
	\item 分类:无定形硅、晶体硅(结构类似金刚石,原子晶体)
	\item 灰黑色晶状固体
	\item 质地较脆
	\item 半导体
\end{itemize}
\subsubsection{化学性质}
\paragraph{与非金属单质反应} 
	\begin{itemize}
		\item $\ce{Si + O2 ->[{高温}] SiO2}$
		\item $\ce{Si + 2Cl2 ->[\Delta] SiCl4}$
		\item $\ce{Si + 2F2 -> SiF4}$
		\item $\ce{Si + C ->[{高温}] \underset{\text{金刚砂}}{SiC}}$
	\end{itemize}
\paragraph{与水反应}
$\underbrace{\ce{Si + H2O + 2NaOH -> Na2SiO3 + 2H2 ^}}_{\text{野外制氢}}$
\paragraph{精炼}
\begin{enumerate}
	\item $\ce{Si + Cl2 ->[\Delta] SiCl4}$
	\item $\ce{SiCl4 + 2H2 ->[{高温}] 4HCl + Si}$
\end{enumerate}
	
\subsection{硅的氧化物}
最简式:\ce{SiO2}(分子晶体)
\subsubsection{物理性质}
\begin{itemize}
	\item 透明、硬度大、熔点高
\end{itemize}
\subsubsection{化学性质}
\paragraph{酸性氧化物}
\subparagraph{与强碱反应}
$\underbrace{\ce{SiO2 + 2NaOH -> Na2SiO3 + H2O}}_{\text{装NaOH溶液不用玻璃塞}}$
\subparagraph{与唯一一种酸氢氟酸反应}
$\underbrace{\ce{SiO2 + 4HF -> SiF4 ^ + 2 H2O}}_{\text{腐蚀玻璃、玻璃雕花}}$
(气标!气标!!)
\subparagraph{与碱性氧化物反应}
氧化硅与碱性氧化物反应,不与水反应(与水反应产物为硅酸,是沉淀,阻止反应进行)
\begin{itemize}
	\item $\ce{SiO2 + Na2O ->[{高温}] Na2SiO3}$
	\item $\ce{SiO2 + CaO ->[{高温}] CaSiO3}$
\end{itemize}
\subparagraph{与碱性盐反应}
\begin{itemize}
	\item $\underbrace{\ce{SiO2 + Na2CO3 ->[{高温}] Na2SiO3 + CO2 ^}}_{\text{制作玻璃}}$
	\item $\underbrace{\ce{SiO2 + CaCO3 ->[{高温}] CaSiO3 + CO2 ^}}_{\text{造渣反应}}$
\end{itemize}
\subparagraph{与碳反应}
\begin{itemize}
	\item $\ce{SiO2 + 2C ->[{高温}] Si + 2CO ^}$
	\item $\ce{SiO2 + 3C ->[{高温}] SiC + 3CO ^}$
\end{itemize}
\subparagraph{精炼}
\begin{enumerate}
	\item $\ce{SiO2 + 4Mg ->[{高温}] Mg2Si + 2MgO}$
	\item $\ce{Mg2Si + 4HCl -> 2MgCl2 + SiH4 ^}$
	\item $\ce{SiH4 + 2O2 -> SiO2 + 2H2O}$(自然)
\end{enumerate}


\subsection{硅的水化物(硅酸、原硅酸)}
硅酸:\ce{H2SiO3}、、
原硅酸:\ce{H4SiO4}
\subsubsection{物理性质}
白色胶状沉淀
\subsubsection{化学性质}
\paragraph{弱酸性}
不使酸碱指示剂变色
\subparagraph{硅酸电离}
$\left\{\begin{array}{lr}
	\ce{H2SiO3 <=> H+ + HSiO3-}\\
	\ce{H2SiO3- <=> H+ + SiO3^2-}\\
\end{array}\right.$
\subparagraph{原硅酸电离}
$\left\{\begin{array}{lr}
	\ce{H4SiO4 <=> H+ + H3SiO4-}\\
	\ce{H3SiO4- <=> H+ + H2SiO4^2-}\\
	\ce{H2SiO4- <=> H+ + HSiO4^3-}\\
	\ce{HSiO4- <=> H+ + SiO4^4-}\\
\end{array}\right.$
\paragraph{不稳定沉淀}
\begin{itemize}
	\item $\ce{H4SiO4 -> H2SiO3 + H2 ^}$
	\item $\ce{H2SiO3 ->[\Delta] SiO2 + H2O}$
	\item $\ce{H2SiO3 ->[\Delta] SiO2*xH2O + H2O}$
\end{itemize}
\paragraph{与强碱反应}
\subparagraph{与氢氧化钠反应}
$\ce{H2SiO3 + 2NaOH -> Na2SiO3 + 2H2O}$
\subparagraph{不与氨气反应}
$\ce{SiO3^2- + 2NH4+ -> H2SiO3 v + 2NH3 ^}$
\subsubsection{制备}
\paragraph{\ce{SiO2}无法一步变成\ce{H2SiO3}}
$\left\{\begin{array}{lr}
	\ce{SiO2 + 2NaOH -> Na2SiO3 + H2O}\\
	\ce{Na2SiO3 + 2HCl -> 2NaCl + H2SiO3 v}\\
\end{array}\right.$

\subsection{硅酸盐}
\subsubsection{物理性质}
\ce{K2SiO3}和\ce{Na2SiO3}溶于水,其余硅酸盐微溶于水。
\subsubsection{化学性质}
\begin{itemize}
	\item $\left\{\begin{array}{lr}
				\ce{Na2SiO3 + CO2 + H2O -> Na2CO3 + H2SiO3 v}\\
				\ce{Na2SiO3 + 2CO2 + 2H2O -> 2NaHCO3 + H2SiO3 v}\\
			\end{array}\right.$
	\item $\left\{\begin{array}{lr}
				\ce{Na2SiO3 + 6HF -> SiF4 ^ + 2NaF + 3H2O}\\
				\underbrace{\ce{CaSiO3 + 6HF -> SiF4 ^ + CaF2 + 3H2O}}_{\text{产物硅酸不稳定生成\ce{SiO2},继续与氢氟酸反应}}\\
			\end{array}\right.$
\end{itemize}
\subsubsection{硅酸盐的拆分}
$活泼金属氧化物\longrightarrow 较活泼金属氧化物\longrightarrow 二氧化硅\longrightarrow 水$
\begin{itemize}
	\item \ce{Na2SiO3}:\ce{Na2O*SiO2}
	\item \ce{CaSiO3}:\ce{CaO*SiO2}
	\item \ce{Al2(Si2O5)(OH)4)}:\ce{Al2O3*2SiO2*2H2O}
\end{itemize}

\subsection{用途与俗称}
\subsubsection{用途}
\begin{itemize}
	\item \ce{Si}(不透明):硅芯片、太阳能电池板
	\item \ce{SiO2}(透明):玻璃、石英玻璃、硅胶(\ce{mSiO2*nH2O},干燥剂)、光导纤维
	\item \ce{SiO3^2-}盐:水泥、陶瓷、防火材料等无机非金属材料
	\item \ce{H2SiO3}:制硅胶
\end{itemize}
\subsubsection{俗称}
\begin{itemize}
	\item \ce{SiO2}:水晶、玛瑙、石英
	\item \ce{Na2SiO3}水溶液:水玻璃
	\item \ce{Na2SiO3}:泡花碱
\end{itemize}