\clearpage
\section{Cl}
\subsection*{氯相关}
\subsubsection{含氯酸}
从上至下,酸性递增,氧化性递减。
\begin{itemize}
	\item \ce{HClO}:次氯酸
	\item \ce{HClO2}:亚氯酸
	\item \ce{HClO3}:氯酸
	\item \ce{HClO4}:高氯酸
\end{itemize}
\subsubsection{卤素}
\begin{itemize}
	\item \ce{F}:无正价,氧化性最强的单质
	\item \ce{Cl}:黄绿色气体
	\item \ce{Br}:常温下唯一液态非金属单质,保存液溴需水封,海水元素
	\item \ce{I}:易升华
\end{itemize}
\begin{itemize}
	\item \ce{AgF}:可溶于水
	\item \ce{AgCl}:白色沉淀
	\item \ce{AgBr}:淡黄色沉淀
	\item \ce{AgI}:黄色沉淀,用于人工降雨
\end{itemize}
\paragraph{海水提溴}
$$
\ce{{海水} ->[{晒盐}] {盐卤} ->[{通入}Cl2] Br2(aq) ->[{吹热空气或水蒸气}] Br2(g) ->[{热饱和纯碱}] {溴酸盐和溴盐溶液} ->[{稀硫酸酸化}] {Br2}}
$$
$\left\{\begin{array}{lr}
	\ce{Cl2 + 2Br- -> Br2 + 2Cl-}\\
	\ce{3Br2 + 3CO3^2- -> 5Br- + BrO3- + 3CO2}\\
	\ce{5Br- + BrO3- + 6H+ -> 3Br2 + 3H2O}\\
\end{array}\right.$
\paragraph{海带提碘}
$$
\ce{{海带} -> {烧碱灰} ->[{泡水浸取}] ->[Cl2] I2}
$$
\paragraph{拟卤素}
$\underset{\text{氰}}{\ce{CN}}$、$\underset{\text{硫氰}}{\ce{SCN}}$、$\underset{\text{氧氰}}{\ce{OCN}}$

\subsection{盐酸}
\subsubsection{物理性质}
无色、有刺激性气味液体。
\subsubsection{化学性质}
\paragraph{酸性}
产物中有盐
\begin{itemize}
	\item $\ce{2H+ + Fe -> Fe^2+ + H2 ^}$
	\item $\ce{H+ + OH- -> H2O}$
	\item $\ce{2H+ + CaO -> Ca2+ + H2O}$
	\item $\ce{2H+ + CO3^2- -> CO2 ^ + H2O}$
\end{itemize}
\paragraph{氧化性}
盐酸的氧化性由\ce{H+}体现
\begin{itemize}
	\item $\ce{2H+ + Fe -> Fe^2+ + H2 ^}$
\end{itemize}
\paragraph{还原性}
\begin{itemize}
	\item $\underbrace{\ce{4HCl({浓}) + MnO2 ->[\Delta] MnCl2 + CL2 ^ + 2H2O}}_{\text{实验室制氯气}}$
	\item $\left\{\begin{array}{lr}
			\ce{16HCl + 2KMnO4 -> 2KCl + 5Cl2 ^ + 2MnCl2 + 8H2O}\\
			\ce{14HCl + K2Cr2O7 -> 2KCl + 3Cl2 ^ + 2CrCl3 + 7H2O}\\
			\ce{6HCl + KClO3 -> KCl + 3Cl2 ^ + 3H2O}\\
			\ce{14HCl + PbO2 -> PbCl2 + Cl2 ^ + 2H2O}\\
			\ce{6HCl + NaBiO3 -> NaCl + Cl2 ^ + BiCl2 + 3H2O}\\
		\end{array}\right.$
\end{itemize}
\subsubsection{制备}
\paragraph{工业}
\begin{enumerate}
	\item $\ce{2NaCl + 2H2O ->[{通电}] 2NaOH + H2 ^ + Cl2 ^}$
	\item $\ce{H2 + Cl2 ->[{点燃}] 2HCl}$
\end{enumerate}
\paragraph{实验室}
\begin{itemize}
	\item $\ce{NaCl + H2SO4({浓}) ->[\Delta] NaHSO4 + HCl ^}$
	\item $\ce{2NaCl + H2SO4({浓}) ->[\Delta] Na2SO4 + 2HCl ^}$
\end{itemize}
\subsection{氯气}
\subsubsection{物理性质}
\begin{itemize}
	\item \textcolor[rgb]{0.745,0.752,0.317}{黄绿色}气体
	\item 密度大于空气,加压易液化
	\item 难溶于饱和食盐水,可溶于水,易溶于\ce{CCl4}。
\end{itemize}
\subsubsection{化学性质}
\paragraph{助燃性}
强氧化性
\begin{itemize}
	\item $\ce{H2 + Cl2 ->[{点燃}] 2HCl}$(苍白色火焰)
	\item $\ce{2Fe + 3Cl2 ->[{点燃}] 2FeCl3}$(产物是三价铁)
	\item $\ce{Cu + Cl2 ->[{点燃}] CuCl2}$
	\item $\ce{2Na + Cl2 ->[{点燃}] 2NaCl}({白烟黄光})$
	\item 磷在氯气中燃烧产生白色烟雾$\left\{\begin{array}{lr}
			\ce{2P + 5Cl2 ->[{点燃}] 2PCl5}({烟})\\
			\ce{2P + 3Cl2 ->[{点燃}] 2PCl3}({雾})\\
		\end{array}\right.$
	\item $\left\{\begin{array}{lr}
			\ce{PCl3 + 3H2O -> H3PO3 + 3HCl}\\
			\ce{PCl5 + 4H2O -> H3PO4 + 5HCl}\\
		\end{array}\right.$
\end{itemize}
\paragraph{氧化性和还原性}
\begin{itemize}
	\item $\left\{\begin{array}{lr}
			\ce{H2O + Cl2 <=> HCl + HClO}\\
			\ce{H2O + Cl2 <=> H+ + Cl- + HClO}\\
		\end{array}\right.$
	\item $\left\{\begin{array}{lr}
			\ce{NaOH + Cl2 -> NaCl + \underset{\text{84消毒液、漂白粉}}{NaClO} + H2O}\\
			\ce{2Ca(OH)2 + 2Cl2 -> CaCl2 + \underset{\text{漂白精、漂白粉}}{Ca(ClO)2} + 2H2O}\\
		\end{array}\right.$
	\item $\left\{\begin{array}{lr}
			\ce{6NaOH + 3Cl2 ->[\Delta] 5NaCl + NaClO3 + 3H2O}\\
			\ce{6KOH + 3Cl2 ->[\Delta] 5KCl + KClO3 + 3H2O}\\
		\end{array}\right.$
	\item $\ce{2H2O + Cl2 + SO2 -> HCl + H2SO4}$
\end{itemize}
\subsubsection{制备}
\paragraph{工业}
\begin{itemize}
	\item $\ce{2NaCl + 2H2O ->[{通电}] 2NaOH + H2 ^ + Cl2 ^}$
	\item $\ce{2NaCl(l) ->[{通电}] 2Na + Cl2 ^}$
\end{itemize}
\paragraph{实验室}
\begin{itemize}
	\item $\ce{MnO2 + 4HCl({浓}) ->[\Delta] Cl2 ^ + MnCl2 + 2H2O}$
\end{itemize}
\subsubsection{除杂}
\begin{itemize}
	\item \ce{Cl2}(\ce{HCl}):饱和食盐水(溶液度:\ce{HCl} > \ce{NaCl} > \ce{Cl2})
	\item \ce{HCl}(\ce{Cl2}):\ce{CCl4}
	\item \ce{CO2}(\ce{HCl}):饱和\ce{NaHCO3}溶液
\end{itemize}
\subsubsection{氯水}
\paragraph{成分}
\begin{itemize}
	\item 分子:\ce{H2O}、\ce{Cl2}、\ce{HClO}
	\item 离子:\ce{Cl-}、\ce{H+}、\ce{ClO-}、\ce{OH-}
\end{itemize}
\paragraph{检验}
\begin{itemize}
	\item \ce{Cl2}:\ce{FeCl2}溶液由\textcolor[rgb]{0.625,0.8,0.7}{浅绿色}变为\textcolor[rgb]{0.835,0.611,0.247}{棕黄色}
	\item \ce{Cl-}:加入硝酸酸化的\ce{AgNO3}溶液,产生白色沉淀
	\item \ce{HClO}:有色布条褪色
	\item \ce{H+}:pH试纸先变红,再褪色
\end{itemize}
\subsubsection{鉴别}
湿润淀粉碘化钾试纸变为\textcolor[rgb]{0.556,0.827,0.898}{蓝色}
$\ce{Cl2 + 2KI -> 2KCl + I2}$
\subsection{次氯酸}
\paragraph{化学式}
\ce{HClO}
\subsubsection{化学性质}
\paragraph{见光分解}
$\ce{2HClO ->[{光}] 2HCl + O2 ^}$
\paragraph{酸性}
\ce{H2CO3} > \ce{HClO} > \ce{HCO3-}
\paragraph{氧化性}
$\ce{HClO + SO2 + H2O -> HCl + H2SO4}$
\subsection{含氯酸盐}
\subsubsection{\ce{NaClO}}
\paragraph{次氯酸钠的变质}
$\left\{\begin{array}{lr}
	\ce{CO2 + NaClO + H2O -> HClO + NaHCO3}\\
	\ce{2HClO ->[{光}] 2HCl + O2 ^}\\
\end{array}\right.$
\paragraph{\ce{SO2}通入\ce{NaClO3}溶液}
$\ce{ClO- + SO2 + H2O -> Cl- + 2H+ + SO4^2-}$
\subsubsection{\ce{Ca(ClO)2}}
\paragraph{次氯酸钙的变质}
$\left\{\begin{array}{lr}
	\ce{CO2 + Ca(ClO)2 + H2O -> 2HClO + CaCO3 v}\\
	\ce{2HClO ->[{光}] 2HCl + O2 ^}\\
\end{array}\right.$
\paragraph{\ce{SO2}通入\ce{Ca(ClO3)2}溶液}
$\ce{Ca^2+ + ClO- + SO2 + H2O -> Cl- + 2H+ + CaSO4 v}$
\subsubsection{\ce{Cl2}逐渐通入\ce{FeI2}和\ce{FeBr2}混合溶液}
\begin{enumerate}
	\item $\ce{Cl2 + 2I- -> 2Cl- + I2}$
	\item $\ce{Cl2 + 2Fe^2+ -> 2Cl- + 2Fe^3+}$
	\item $\ce{Cl2 + 2Br- -> 2Cl- + Br2 ^}$
	\item $\ce{5Cl2 + 6H2O + I2 -> 12H+ + 2IO3- + 10Cl-}$
\end{enumerate}
\subsubsection{\ce{Cl2}逐渐通入\ce{Na2CO}溶液}
\begin{equation}\label{equ:E1}
		\ce{H2O + Cl2 <=> HCl + HClO}
\end{equation}
\begin{equation}\label{equ:E2}
		\ce{HCl + Na2CO3 -> NaCl + NaHCO3}
\end{equation}
\begin{equation}\label{equ:E3}
		\ce{HCl + NaHCO3 -> NaCl + H2O + CO2 ^}
\end{equation}
\begin{equation}\label{equ:E4}
		\ce{HClO + Na2Co3 <=> NaClO + NaHCO3}
\end{equation}
注意\ce{HClO}和\ce{NaHCO3}不反应。
\begin{enumerate}
	\item $\ce{2Na2CO3 + Cl + H2O -> 2NaHCO3 + NaCl + NaClO}$(\ref{equ:E1}+\ref{equ:E2}+\ref{equ:E4})
	\item $\ce{Cl2 + Na2CO3 + H2O -> NaCl + NaHCO3 + HClO}$(\ref{equ:E1}+\ref{equ:E2})
	\item $\ce{Na2CO3 + 2Cl2 + H2O -> CO2 ^ + 2NaCl + 2HClO}$(\ref{equ:E1}+\ref{equ:E2}+\ref{equ:E3})
\end{enumerate}