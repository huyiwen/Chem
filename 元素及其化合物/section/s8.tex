
\clearpage
\section{N}
\subsection{氨气}
\subsubsection{物理性质}
\begin{itemize}
	\item 无色、刺激性气体
	\item 极易溶于水
	\item 加压易液化(制冷剂)
\end{itemize}
\subsubsection{尾气处理防倒吸}
\ce{NH3}或\ce{HCl}等气体极易溶于水,直接通入水中会使水倒吸。在水层下放\ce{CCl4}层并将气体通入,可以防止倒吸。(\ce{NH3}和\ce{HCl}不溶于\ce{CCl4})
\subsubsection{喷泉实验}
\begin{center}
\begin{tabular}{|c|c|}
	\hline
	气体&液体\\\hline
	\ce{NH3}&水或稀\ce{H2SO4}\\\hline
	\ce{HCl}&水或\ce{NaOH}溶液\\\hline
	\ce{Cl2}&~\\\cline{1-1}
	\ce{CO2}&\ce{NaOH}\\\cline{1-1}
	\ce{SO2}&溶液\\\cline{1-1}
	\ce{H2S}&~\\\hline
\end{tabular}
\end{center}
\subsubsection{化学性质}
\paragraph{可燃性}
\begin{itemize}
	\item $\ce{4NH3 + 3O2 ->[\Delta{或点燃}] 2N2 + 6H2O}$
\end{itemize}
\paragraph{碱性}
唯一的碱性气体
\begin{itemize}
	\item $\ce{NH3 + HCl -> \underset{\text{白烟}}{\ce{NH4Cl}}}$
	\item $\ce{NH3 + H2O <=> NH3*H2O <=> NH4+ + OH-}$
\end{itemize}
\paragraph{还原性}
\begin{itemize}
	\item 催化氧化:$\ce{4NH3 + 5O2 <=>[Pt][\Delta] 4NO + 6H2O}$
	\item $\left\{\begin{array}{lr}
			\ce{2NH3 + 3Cl2 -> N2 + 6HCl}\\
			\ce{8NH3 + 3Cl2 -> N2 + 6\underset{\text{白烟:检验氯气泄漏}}{\ce{NH4Cl}}}\\
		\end{array}\right.$
	\item $\ce{2NH3 + CuO ->[\Delta] 3Cu + N2 + 3H2O}$
\end{itemize}
\subsubsection{检验与验满}
\begin{itemize}
	\item 检验:\ce{NH3}能使湿润红色石蕊试纸变蓝(没有紫色石蕊试纸)。
	\item 验满:沾取少量浓盐酸,置于瓶口,出现白烟。
\end{itemize}

\subsubsection{制备}
\begin{itemize}
	\item $\ce{Ca(OH)2 + 2NH4Cl ->[\Delta] CaCl2 + 2NH3 ^ + 2H2O}$
\end{itemize}
\subsubsection{用途}
制硝酸、氮肥、制冷剂

\subsection{氮气}
\subsubsection{物理性质}
\begin{itemize}
	\item 无色无味气体、难溶于水
\end{itemize}
\subsubsection{化学性质}
氮气常温下不活泼(氮氮三键)、高温下(氮原子)活泼。
\begin{itemize}
	\item $\ce{N2 + 3Mg ->[{点燃}] \underset{\text{\textcolor[rgb]{0.874,0.890,0.756}{淡黄色}}}{Mg3N2}}$
	\item $\ce{N2 + 3H2 <=>[{高温、高压}][{催化剂}] 2NH3}$
	\item $\ce{N2 + O2 ->[{高温}] 2NO}$
\end{itemize}
\subsubsection{制备}
\begin{itemize}
	\item $\ce{NaNO2 + NH4Cl ->[\Delta] NaCl + N2 ^ + 2H2O}$
\end{itemize}

\subsection{氮的氧化物}
\subsubsection{物理性质}
\begin{itemize}
	\item \ce{NO}:无色气体、有毒、难溶于水
	\item \ce{NO2}:\textcolor[rgb]{0.827,0.286,0.184}{红棕色}气体、有毒、与水反应
	\item \ce{N2O4}:无色气体、有毒、与水反应、化学性质类似\ce{NO2}、标况非气体
\end{itemize}
\subsubsection{化学性质}
\paragraph{一些实际发生的反应}
\begin{itemize}
	\item $\ce{2NO + O2 -> NO2}$(迅速转变为\textcolor[rgb]{0.827,0.286,0.184}{红棕色})
	\item $\ce{3NO2 + H2O -> 2HNO3 + NO}$(歧化)
	\item $\ce{2NO2 <=> N2O4}$
	\item $\ce{2NO2 + 2NaOH -> NaNO2 + NaNO3 + H2O}$
	\item $\ce{NO + NO2 + 2NaOH -> 2NaNO3 + H2O}$
\end{itemize}
\paragraph{推导反应(只能用于计算)}
\begin{itemize}
	\item $\ce{3NO2 + H2O -> 2HNO3 + NO}$
	\item $\ce{4NO2 + O2 + 2H2O -> 4HNO3}$
	\item $\ce{4NO + 3O2 + 2H2O -> 4HNO3}$
\end{itemize}
\paragraph{与氮的氢化物反应}
\begin{itemize}
	\item $\ce{6NO + 4NH3 ->[\Delta] 5N2 + 6H2O}$
	\item $\ce{6NO2 + 8NH3 ->[\Delta] 7N2 + 12H2O}$
	\item $\underbrace{\ce{N2O4 + 3N2H4 ->[\Delta] 3N2 + 4H2O}}_{\text{火箭推进}}$
\end{itemize}
\subsubsection{酸酐}
将可电离的\ce{H+}配合\ce{O}分解。
$$
\ce{H2SO4 -> SO3 + H2O}\\
\ce{2HNO3 -> N2O5 + H2O}\\
$$
\paragraph{化学性质}
与碱反应生成盐和水
\paragraph{与酸性氧化物的关系}
酸酐是酸性氧化物或非氧化物,酸性氧化物一定是酸酐。

\subsection{硝酸}
	\subsubsection{物理性质}
\begin{itemize}
	\item 无色、有刺激性气味
\end{itemize}
\subsubsection{化学性质}
\paragraph{氧化性}
活泼金属与硝酸反应时不生成氢气。
\begin{itemize}
	\item $\left\{\begin{array}{lr}
			\ce{Cu + 4HNO3({浓}) -> Cu(NO3)2 + 2NO2 ^ + 2H2O}\\
			\ce{Cu + 8HNO3({稀}) -> 3Cu(NO3)2 + 2NO ^ + 4H2O}\\
		\end{array}\right.$
	\item $\left\{\begin{array}{lr}
			\ce{Zn + 4HNO3({浓}) -> Zn(NO3)2 + 2NO2 ^ + 2H2O}\\
			\ce{Zn + 8HNO3({稀}) -> 3Zn(NO3)2 + 2NO ^ + 4H2O}\\
			\ce{4Zn + 10HNO3({更稀}) -> 4Zn(NO3)2 + N2O ^ + 5H2O}\\
			\ce{4Zn + 10HNO3({极稀}) -> 4Zn(NO3)2 + NH4NO3 + 3H2O}\\
		\end{array}\right.$
	\item $\ce{C + 4HNO3({浓})  ->[\Delta] 4NO2 ^ + CO2 ^ + 2H2O}$
\end{itemize}
\paragraph{不稳定性}
\begin{itemize}
	\item $\ce{4HNO3  ->[\Delta] 4NO2 ^ + O2 ^ + 2H2O}$
\end{itemize}
\paragraph{漂白性}
浓硝酸可以漂白石蕊溶液
\subsubsection{制备}
\begin{enumerate}
	\item $\ce{N2 + 3H2 <=>[{高温、高压}][{催化剂}] 2NH3}$
	\item $\ce{4NH3 + 5O2 <=>[Pt][\Delta] 4NO + 6H2O}$(催化剂一明一暗)
	\item $\ce{2NO + O2 -> 2NO2}$
	\item $\ce{3NO2 + H2O -> 2HNO3 + \underset{\text{雾}}{NO}}$
	\item ($\ce{HNO3 + NH3 -> \underset{\text{烟}}{NH4NO3}}$)
\end{enumerate}
装置:硬质石英玻璃\\
现象:催化剂一明一暗,有\textcolor[rgb]{0.827,0.286,0.184}{红棕色}气体和白色烟雾生成。
\subsubsection{固氮}
\paragraph{固氮}
将游离态的氮(氮气)转化为化合态的氮
\paragraph{自然固氮}
\subparagraph{高能固氮}
雷雨发庄稼
\begin{enumerate}
	\item $\ce{N2 + O2 ->[{放电}] 2NO}$
	\item $\ce{2NO + O2 -> 2NO2}$
	\item $\ce{3NO2 + H2O -> 2HNO3 + NO}$
\end{enumerate}
\subparagraph{生物固氮}
大豆根瘤菌
\paragraph{人工固氮}
合成氨
	
\subsection{盐}
\subsubsection{硝酸盐分解规律}
\begin{itemize}
	\item K到Mg:亚硝酸盐和氧气($\ce{2NaNO3 ->[\Delta] 2NaNO2 + O2 ^}$)
	\item Al到Cu:金属氧化物、二氧化氮和氧气($\ce{2Pb(NO3)2 ->[\Delta] 2PbO + 4NO2 ^ + O2 ^}$)
	\item Hg到Ag:金属单质、二氧化氮和氧气($\ce{2AgNO3 ->[\Delta] 2Ag + 2NO2 + O2 ^}$)
\end{itemize}
\subsubsection{铵盐分解规律}
\begin{itemize}
	\item $\ce{NH4NO3 ->[\Delta] N2O ^ + 2H2O}$
	\item $\ce{NH4HCO3 ->[\Delta] NH3 ^ + CO2 ^ + H2O}$
	\item $\ce{NH4Cl ->[\Delta] N2O ^ + HCl ^}$
	\item $\ce{(NH4)2Cr2O7 ->[\Delta] N2 ^ + CrO3 + 4H2O}$
\end{itemize}