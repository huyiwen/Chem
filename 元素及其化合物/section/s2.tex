\clearpage


\section{\ce{Na}}


\subsection{\ce{Na}单质}

\subsubsection{物理性质}

\begin{itemize}
	\item 银白色固体,有金属性光泽
	\item 密度介于水和煤油之间,用煤油或石蜡保存
	\item 熔点低
	\item 质地较软,可以用小刀切割
\end{itemize}

\subsubsection{化学性质}

\paragraph{与非金属单质反应}

\begin{itemize}
	\item $\left\{\begin{array}{lr}
			\ce{4Na + O2 -> 2Na2O}\\
			\ce{2Na + O2 ->[\Delta] Na2O2}\\
		\end{array}\right.$
	\item $\ce{2Na + S -> Na2S}$
	\item $\ce{2Na + H2 ->[\Delta] 2NaH}$
	\item $\left\{\begin{array}{lr}
			\ce{2Na + Br2 -> 2NaBr}\\
			\ce{2Na + Cl2 ->[{点燃}] 2NaCl}\\
		\end{array}\right.$
\end{itemize}

\paragraph{与水反应}

$\ce{2Na + 2H2O -> 2NaOH + H2 ^}$

\begin{description}
	\item{浮} 钠的密度比水小
	\item{溶} 反应放热,钠的熔点低
	\item{游} 生成氢气推动钠
	\item{响} 反应剧烈
	\item{红} 生成\ce{NaOH}遇到酚酞变红
\end{description}

\paragraph{与盐酸反应}

$\ce{2Na + 2HCl -> 2NaCl + H2 ^}$

\paragraph{与碱反应}

实质是先与水反应,产物再和盐反应。

\paragraph{与盐溶液反应}

实质是先与水反应,产物再和盐反应(钠不能与盐溶液发生置换反应)。

\begin{itemize}
	\item 钠与硫酸铜溶液
	$\left\{\begin{array}{lr}
		\ce{2Na + 2H2O -> 2NaOH + H2 ^}\\
		\ce{2NaOH + CuSO4 -> Na2SO4 + Cu(OH)2 v}\\
	\end{array}\right.$
\end{itemize}

\paragraph{与\ce{CO2}反应}

$\left\{\begin{array}{lr}
	\ce{4Na + CO2 ->[\Delta] 2Na2O + C}\\
	\ce{4Na + 3CO2 ->[\Delta] 2Na2CO3 + C}\\
		\end{array}\right.$

\subsubsection{钠的制取}

$\left\{\begin{array}{lr}
	\ce{2NaCl(l) ->[{电解}] 2Na + Cl2 ^}\\
	\ce{2NaOH(l) ->[{电解}] 2Na + O2 ^ + H2 ^}\\
\end{array}\right.$

\subsubsection{钠的用途}

\begin{itemize}
	\item 冶炼金属:$\ce{4Na + TiCl4(l) -> 4NaCl + Ti}$
	\item 原子反应导热剂
	\item 钠光灯
\end{itemize}

\subsubsection{焰色反应}

钠盐:黄色火焰% \textcolor[rgb]{0.964,0.913,0.313}{黄色}火焰


\subsection{\ce{Na}的化合物}

\subsubsection{氧化钠和过氧化钠}

\paragraph{比较氧化钠和过氧化钠}

\begin{center}
\begin{tabular}{|c|c|c|}
	\hline
	名称&氧化钠&过氧化钠\\\hline
	化学式&\ce{Na2O}&\ce{Na2O2}\\\hline
	物理性质&白色固体&\textcolor[rgb]{0.968,0.898,0.686}{淡黄色}固体\\\hline
	氧化物类型&碱性氧化物&过氧化物\\\hline
	获取&$\ce{4Na + O2 -> 2Na2O}$&$\ce{2Na + O2 ->[\Delta] Na2O2}$\\\hline
	与水反应&$\ce{Na2O + H2O -> 2\underset{\text{白色粘稠物}}{\ce{NaOH}}}$&$\ce{2Na2O2 + 2H2O -> 4\underset{\text{白色粘稠物}}{\ce{NaOH}} + O2 ^}$\\\hline
	与酸反应&$\ce{Na2O + 2H+ -> 2Na+ + H2O}$&$\ce{2Na2O2 + 4H+ -> 4Na+ + 2H2O + O2 ^}$\\\hline
	与\ce{CO2}反应&$\ce{Na2O + CO2 -> Na2CO3}$&$\ce{2Na2O2 + 2CO2 -> 2Na2CO3 + O2}$\\\hline
	用途&制取烧碱&漂白剂、消毒剂、供氧剂\\\hline
\end{tabular}
\end{center}

\paragraph{过氧化钠的强氧化性}

\begin{itemize}
	\item 与\ce{SO2}反应:$\ce{Na2O2 + SO2 -> Na2SO4}$
	\item 投入\ce{FeCl2}溶液中生成\ce{Fe(OH)3}沉淀
	\item 投入氢硫酸,氧化硫化氢成硫单质,溶液浑浊
	\item 氧化\ce{SO3^2-}成\ce{SO4^2-}
	\item 使品红溶液褪色
\end{itemize}

\subsubsection{碳酸钠和碳酸氢钠}

\paragraph{碳酸钠\ce{Na2CO3}}

\begin{itemize}
	\item 俗名:纯碱、苏打
	\item 与盐酸反应:$\ce{Na2CO3 + 2HCl -> 2NaCl + H2O + CO2 ^}$
	\item 与\ce{Ca(OH)2}溶液反应:$\ce{Na2CO3 + Ca(OH)2 -> CaCO3 v + 2NaOH}$
	\item 与\ce{BaCl2}溶液反应:$\ce{Na2CO3 + BaCl2 -> BaCO3 v + 2NaCl}$
\end{itemize}

\paragraph{碳酸氢钠\ce{NaHCO3}}

\begin{itemize}
	\item 俗名:小苏打
	\item 与盐酸反应:$\ce{NaHCO3 + HCl -> NaCl + H2O + CO2 ^}$
	\item 与过量\ce{Ca(OH)2}溶液反应:$\ce{Ca2+ + OH- + HCO3- -> CaCO3 v + H2O}$
	\item 与少量\ce{Ca(OH)2}溶液反应:$\ce{Ca2+ + 2OH- + 2HCO3- + Ca(OH)2 -> CaCO3 v + 2H2O + CO3^2-}$
	\item 与\ce{BaCl2}溶液反应:无明显现象
	\item 受热分解:$\ce{2NaHCO3 ->[\Delta] Na2CO3 + H2O + CO2 ^}$
\end{itemize}

\paragraph{相互转换}

$\ce{Na2CO3 <=>[CO2 + H2O{或}H+][\Delta({固体}){或}OH-] NaHCO3}$

\paragraph{鉴别\ce{Na2CO3}和\ce{NaHCO3}}

\subparagraph{固体}

根据热稳定性加热,能产生使澄清石灰水变浑浊的气体的是\ce{NaHCO3}

\subparagraph{溶液}

\begin{itemize}
	\item 与可溶性钙、钡盐生成沉淀的是\ce{Na2CO3}
	\item 与足量盐酸反应剧烈的是\ce{NaHCO3}
	\item 逐滴加盐酸先生成气体的是\ce{NaHCO3}
	\item 等物质的量pH值较大的是\ce{Na2CO3}
\end{itemize}
