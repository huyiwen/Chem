%%%%%%%%%%%%%%%%%%%%%%%%%%%%%%%%%%%%%%%%%%%%%%%%%%%%%%%%%%%%
% 源文件排版规范:
% chapter分文件放置于子目录下,需要clearpage
% section前空三行
% subsection前空两行
% subsubsection及以下级别前空一行
%%%%%%%%%%%%%%%%%%%%%%%%%%%%%%%%%%%%%%%%%%%%%%%%%%%%%%%%%%%%
\documentclass[10pt]{article}
\usepackage{xeCJK}                      % xeCJK
\usepackage{fontspec}                   % 字体选择
\usepackage{fontenc}                    % 字体选择
\usepackage{cite}                       % 引用
\usepackage{natbib}                     % Bibtex引用
\usepackage{indentfirst}                % 首行缩进
\usepackage{hyperref}                   % 目录链接
\usepackage{geometry}                   % 页面调整
\usepackage{chemfig}                    % 化学
\usepackage[version=4]{mhchem}          % 化学
%\usepackage{amsmath}                   % 数学
%\usepackage{amssymb}                   % 数学
%\usepackage{calc}                      % 数学
\usepackage{siunitx}                    % 数学
\usepackage{supertabular}               % 超级表格
\usepackage{multirow}                   % 合并单元格
%%%%%%%%%%%%%%%%%%%%%%%%%%%%%%%%%%%%%%%%%%%%%%%%%%%%%%%%%%%%
\geometry{                              % 页面调整B5活页纸
	dvipdfm,
	left=2.1cm,
	right=2.1cm,
	top=1cm,
	bottom=1.5cm,
	body={14cm,20cm},
	papersize={18.2cm,22.5cm}
}
\hypersetup{                            % 链接颜色调整
    colorlinks,
    citecolor=black,
    filecolor=black,
    linkcolor=black,
    urlcolor=black
}
\setCJKmainfont{Hiragino Sans GB}       % 中文字体
\setmainfont{Times New Roman}           % 西文字体
\XeTeXlinebreaklocale "zh"              % 中文自动换行
\XeTeXlinebreakskip = 0pt plus 1pt
\setlength{\parindent}{2em}             % 首行缩进
\linespread{1.2}                        % 行间距
\setlength{\parskip}{0.5\baselineskip}  % 段间距
\renewcommand\arraystretch{1.5}         % 表格高度调整
\setcounter{tocdepth}{3}                % 目录深度(打印设为1)
%%%%%%%%%%%%%%%%%%%%%%%%%%%%%%%%%%%%%%%%%%%%%%%%%%%%%%%%%%%%
\renewcommand\contentsname{目录}
\renewcommand\bibname{参考文献}
\title{元素及其化合物}
\author{胡译文}
\date{}
%%%%%%%%%%%%%%%%%%%%%%%%%%%%%%%%%%%%%%%%%%%%%%%%%%%%%%%%%%%%
\begin{document}

	\maketitle
	\tableofcontents
	%\clearpage



\section{Flashback}

\subsection{氧化性顺序}

\paragraph{氧化性}

\ce{F2} > \ce{O2} > \ce{Cl2} > \ce{Br2} > \ce{Fe^3+} > \ce{I2} > \ce{S}

\ce{F-} < \ce{H2O} < \ce{Cl-} < \ce{Br-} < \ce{Fe^2+} < \ce{I-} < \ce{S^2-} < \text{惰性电极}

\paragraph{还原性}

\subparagraph{金属活动顺序表}

\begin{itemize}
	\item 钾钙钠镁铝:\ce{K} Ca Na Mg Al
	\item 锌铁锡铅氢:Zn Fe Sn Pb H
	\item 铜汞银铂金:Cu Hg Ag Pt Au
\end{itemize}

\ce{K} > \ce{Ca} > \ce{Na} > \ce{Mg} > \ce{Al} > \ce{Zn} > \ce{Fe} > \ce{Sn} > \ce{Pb} > \ce{H+} > \ce{Cu} > \ce{Hg} > \ce{Ag} > \ce{Pt} > \ce{Au}

\ce{K+} < \ce{Ca^2+} < \ce{Na+} < \ce{Mg^2+} < \ce{Al^3+} < \ce{H2O} < \ce{Zn^2+} < \ce{Fe^2+} < \ce{Sn^2+} < \ce{Pb^2+} < \ce{H+} < \ce{Cu+} < \ce{Hg+} < \ce{Fe^3+} < \ce{Ag+} < \ce{AuCl4-}

\subsection{元素氧化图解}

\begin{figure}[h]
	\centering
	\includegraphics[scale=0.8]{res/Redox.pdf}
\end{figure}

	%\clearpage


\section{\ce{Na}}


\subsection{\ce{Na}单质}

\subsubsection{物理性质}

\begin{itemize}
	\item 银白色固体,有金属性光泽
	\item 密度介于水和煤油之间,用煤油或石蜡保存
	\item 熔点低
	\item 质地较软,可以用小刀切割
\end{itemize}

\subsubsection{化学性质}

\paragraph{与非金属单质反应}

\begin{itemize}
	\item $\left\{\begin{array}{lr}
			\ce{4Na + O2 -> 2Na2O}\\
			\ce{2Na + O2 ->[\Delta] Na2O2}\\
		\end{array}\right.$
	\item $\ce{2Na + S -> Na2S}$
	\item $\ce{2Na + H2 ->[\Delta] 2NaH}$
	\item $\left\{\begin{array}{lr}
			\ce{2Na + Br2 -> 2NaBr}\\
			\ce{2Na + Cl2 ->[{点燃}] 2NaCl}\\
		\end{array}\right.$
\end{itemize}

\paragraph{与水反应}

$\ce{2Na + 2H2O -> 2NaOH + H2 ^}$

\begin{description}
	\item{浮} 钠的密度比水小
	\item{溶} 反应放热,钠的熔点低
	\item{游} 生成氢气推动钠
	\item{响} 反应剧烈
	\item{红} 生成\ce{NaOH}遇到酚酞变红
\end{description}

\paragraph{与盐酸反应}

$\ce{2Na + 2HCl -> 2NaCl + H2 ^}$

\paragraph{与碱反应}

实质是先与水反应,产物再和盐反应。

\paragraph{与盐溶液反应}

实质是先与水反应,产物再和盐反应(钠不能与盐溶液发生置换反应)。

\begin{itemize}
	\item 钠与硫酸铜溶液
	$\left\{\begin{array}{lr}
		\ce{2Na + 2H2O -> 2NaOH + H2 ^}\\
		\ce{2NaOH + CuSO4 -> Na2SO4 + Cu(OH)2 v}\\
	\end{array}\right.$
\end{itemize}

\paragraph{与\ce{CO2}反应}

$\left\{\begin{array}{lr}
	\ce{4Na + CO2 ->[\Delta] 2Na2O + C}\\
	\ce{4Na + 3CO2 ->[\Delta] 2Na2CO3 + C}\\
		\end{array}\right.$

\subsubsection{钠的制取}

$\left\{\begin{array}{lr}
	\ce{2NaCl(l) ->[{电解}] 2Na + Cl2 ^}\\
	\ce{2NaOH(l) ->[{电解}] 2Na + O2 ^ + H2 ^}\\
\end{array}\right.$

\subsubsection{钠的用途}

\begin{itemize}
	\item 冶炼金属:$\ce{4Na + TiCl4(l) -> 4NaCl + Ti}$
	\item 原子反应导热剂
	\item 钠光灯
\end{itemize}

\subsubsection{焰色反应}

钠盐:黄色火焰% \textcolor[rgb]{0.964,0.913,0.313}{黄色}火焰


\subsection{\ce{Na}的化合物}

\subsubsection{氧化钠和过氧化钠}

\paragraph{比较氧化钠和过氧化钠}

\begin{center}
\begin{tabular}{|c|c|c|}
	\hline
	名称&氧化钠&过氧化钠\\\hline
	化学式&\ce{Na2O}&\ce{Na2O2}\\\hline
	物理性质&白色固体&\textcolor[rgb]{0.968,0.898,0.686}{淡黄色}固体\\\hline
	氧化物类型&碱性氧化物&过氧化物\\\hline
	获取&$\ce{4Na + O2 -> 2Na2O}$&$\ce{2Na + O2 ->[\Delta] Na2O2}$\\\hline
	与水反应&$\ce{Na2O + H2O -> 2\underset{\text{白色粘稠物}}{\ce{NaOH}}}$&$\ce{2Na2O2 + 2H2O -> 4\underset{\text{白色粘稠物}}{\ce{NaOH}} + O2 ^}$\\\hline
	与酸反应&$\ce{Na2O + 2H+ -> 2Na+ + H2O}$&$\ce{2Na2O2 + 4H+ -> 4Na+ + 2H2O + O2 ^}$\\\hline
	与\ce{CO2}反应&$\ce{Na2O + CO2 -> Na2CO3}$&$\ce{2Na2O2 + 2CO2 -> 2Na2CO3 + O2}$\\\hline
	用途&制取烧碱&漂白剂、消毒剂、供氧剂\\\hline
\end{tabular}
\end{center}

\paragraph{过氧化钠的强氧化性}

\begin{itemize}
	\item 与\ce{SO2}反应:$\ce{Na2O2 + SO2 -> Na2SO4}$
	\item 投入\ce{FeCl2}溶液中生成\ce{Fe(OH)3}沉淀
	\item 投入氢硫酸,氧化硫化氢成硫单质,溶液浑浊
	\item 氧化\ce{SO3^2-}成\ce{SO4^2-}
	\item 使品红溶液褪色
\end{itemize}

\subsubsection{碳酸钠和碳酸氢钠}

\paragraph{碳酸钠\ce{Na2CO3}}

\begin{itemize}
	\item 俗名:纯碱、苏打
	\item 与盐酸反应:$\ce{Na2CO3 + 2HCl -> 2NaCl + H2O + CO2 ^}$
	\item 与\ce{Ca(OH)2}溶液反应:$\ce{Na2CO3 + Ca(OH)2 -> CaCO3 v + 2NaOH}$
	\item 与\ce{BaCl2}溶液反应:$\ce{Na2CO3 + BaCl2 -> BaCO3 v + 2NaCl}$
\end{itemize}

\paragraph{碳酸氢钠\ce{NaHCO3}}

\begin{itemize}
	\item 俗名:小苏打
	\item 与盐酸反应:$\ce{NaHCO3 + HCl -> NaCl + H2O + CO2 ^}$
	\item 与过量\ce{Ca(OH)2}溶液反应:$\ce{Ca2+ + OH- + HCO3- -> CaCO3 v + H2O}$
	\item 与少量\ce{Ca(OH)2}溶液反应:$\ce{Ca2+ + 2OH- + 2HCO3- + Ca(OH)2 -> CaCO3 v + 2H2O + CO3^2-}$
	\item 与\ce{BaCl2}溶液反应:无明显现象
	\item 受热分解:$\ce{2NaHCO3 ->[\Delta] Na2CO3 + H2O + CO2 ^}$
\end{itemize}

\paragraph{相互转换}

$\ce{Na2CO3 <=>[CO2 + H2O\text{或}H+][\Delta(\text{固体}){或}OH-] NaHCO3}$

\paragraph{鉴别\ce{Na2CO3}和\ce{NaHCO3}}

\subparagraph{固体}

根据热稳定性加热,能产生使澄清石灰水变浑浊的气体的是\ce{NaHCO3}

\subparagraph{溶液}

\begin{itemize}
	\item 与可溶性钙、钡盐生成沉淀的是\ce{Na2CO3}
	\item 与足量盐酸反应剧烈的是\ce{NaHCO3}
	\item 逐滴加盐酸先生成气体的是\ce{NaHCO3}
	\item 等物质的量pH值较大的是\ce{Na2CO3}
\end{itemize}

	\clearpage



\section{镁和铝}


\subsection{铝单质}

\subsubsection{物理性质}

银白色固体、导电性优良(\ce{Ag} > \ce{Cu} > \ce{Al})、熔点低、密度小

\subsubsection{化学性质}

\paragraph{与非金属单质反应}

\begin{itemize}
	\item $\ce{4Al + 3O2 ->[\text{点燃}] \underset{\text{离子晶体}}{\ce{2Al2O3}}}$(铝在氧气中无法剧烈燃烧)
	\item 铝在空气中生成致密的氧化膜,阻止反应;但硝酸汞可以阻止致密的氧化膜生成,剧烈反应,俗称“铝汞齐”。
	\item $\ce{2Al + 3Cl2 ->[\text{点燃}] \underset{\text{分子晶体}}{\ce{2AlCl3}}}$(铝在氯气中可以剧烈燃烧)
	\item $\ce{2Al + N2 ->[\text{高温}] \underset{\text{原子晶体}}{\ce{2AlN}}}$
	\item $\ce{2Al + 3S ->[\Delta] Al2S3}$
\end{itemize}

%\subitem $\ce{2Mg + O2 ->[\text{点燃}] 2MgO}$(耀眼白光)
%\item 与 \ce{CO2}反应:
%\subitem $\ce{2Mg + CO2 ->[\text{点燃}] 2MgO + C}$(耀眼白光,黑色固体生成)
%\subitem $\ce{3Mg + N2 ->[\text{点燃}] Mg3N2}$
%\subitem $\ce{2Mg + Cl2 ->[\text{点燃}] 2MgCl2}$
%\subitem $\ce{Mg + S ->[\Delta] MgS}$
%镁在空气中燃烧时会同时发生前三个反应。

\paragraph{与热水反应}

\begin{itemize}
	\item $\ce{Mg + H2O(\text{沸水}) -> Mg(OH)2 + H2 ^}$
	\item $\ce{2Al + 6H2O -> 2Al(OH)3 + 3H2 ^}$
\end{itemize}

\paragraph{铝(镁)热反应}

可以与 \ce{FeO}、 \ce{Fe2O3}、 \ce{Fe3O4}、 \ce{Cr2O3}、 \ce{MnO2}、 \ce{V2O5}等氧化物反应。用于焊接金属、冶炼难溶金属。

\begin{itemize}
	\item $\ce{2Al + Fe2O3 ->[\text{高温}] Al2O3 + 2Fe}$
	\item $\ce{2Al + Cr2O3 ->[\text{高温}] Al2O3 + 2Cr}$
\end{itemize}

\paragraph{两性}

\begin{itemize}
	\item 与非氧化性酸:$\ce{2Al + 6H+ -> 2Al3+ + 3H2 ^}$
	\item 与氧化性酸:在冷的浓硫酸或浓硝酸中钝化.
	\item 与强碱:$\ce{2Al + 2NaOH + 6H2O -> 2NaAlO2 + 4H2O + 3H2 ^}$
\end{itemize}

\subsubsection{制备}

\paragraph{工业制铝}

$$
\ce{2Al2O3(l) ->[\text{冰晶石}][\text{通电}] 4Al + 3O2 ^}
$$

熔融冰晶石(\ce{Na3AlF6})可以溶解 \ce{Al2O3},是助熔剂,而非催化剂。

\begin{enumerate}
	\item 粉碎
	\item \ce{NaOH}溶液浸泡:$\ce{Al2O3 + 2OH- -> 2AlO2- + H2O}$
	\item 过滤
	\item 通入 \ce{CO2}:$\ce{CO2 + AlO2- + 2H2O -> Al(OH)3 v + HCO3-}$
	\item 过滤
	\item 煅烧:$\ce{2Al(OH)3 ->[\Delta] Ak2O3 + 3H2O}$
	\item 电解:$\ce{2Al2O3(l) ->[\text{冰晶石}][\text{通电}] 4Al + 3O2 ^}$
\end{enumerate}

\paragraph{工业制镁}

\begin{itemize}
	\item $\ce{Mg2+ + 2OH- -> Mg(OH)2 v}$
	\item $\ce{Mg(OH)2 + 2HCl -> MgCl2 + H2O}$
	\item $\ce{MgCl2(l) ->[\text{通电}] Mg + Cl2 ^}$
\end{itemize}

\paragraph{海水提镁}

$$
 \ce{\underset{\text{贝壳}}{CaCO3} -> CaO -> Ca(OH)2 -> Mg(OH)2 -> MgCl2 ->[\text{通电}] Mg}
$$

其中氯元素可以循环:$\ce{Cl2 -> HCl -> MgCl2 -> Cl2}$


\subsection{氧化铝}

\subsubsection{物理性质}

\begin{itemize}
	\item 熔点高、硬度大。
	\item 俗称:刚玉、宝石。
	\item 用途:氧化铝坩锅、装饰品、蓝宝石保护层
\end{itemize}


\subsubsection{化学性质}

\paragraph{两性}

\begin{itemize}
	\item $\ce{Al2O3 + 6H+ -> 2Al^3+ + 3H2O}$
	\item $\ce{Al2O3 + 2OH- -> 2AlO2- + H2O}$
\end{itemize}


\subsection{氢氧化铝}

\subsubsection{化学性质}

\paragraph{两性}

\subparagraph{与强碱反应}

\begin{itemize}
	\item $\ce{2Al + 6H+ -> 2Al^3+ + 3H2 ^}$(非氧化性酸)
	\item $\ce{Al2O3 + 6H+ -> 2Al^3+ + 3H2O}$
	\item $\ce{Al(OH)3 + 3H+ -> Al^3+ + 3H2O}$
\end{itemize}

\subparagraph{与强碱反应}

\begin{itemize}
	\item $\ce{2Al + 2OH- + 2H2O -> 2AlO2- + 3H2 ^}$
	\item $\ce{Al2O3 + 2OH- -> 2AlO2- + H2O}$
	\item $\ce{Al(OH)3 + OH- -> AlO2- + 2H2O}$
\end{itemize}

\subparagraph{ \ce{Al(OH)3}的电离}

\begin{itemize}
	\item $\ce{Al(OH)3 <=> H+ + AlO2- + H2O}$
	\item $\ce{Al(OH)3 <=> Al^3+ + 3OH-}$
\end{itemize}

\paragraph{受热分解}


\subsection{铝离子}

\paragraph{与 \ce{NaOH}的相互滴加}

缓慢滴加并搅拌

\subparagraph{将 \ce{NaOH}滴入 \ce{Al^3+}溶液中}

\begin{enumerate}
	\item 先出现白色沉淀:$\ce{Al^3+ + 3OH- -> Al(OH)3 v}\\$
	\item 后沉淀消失:$\ce{Al(OH)3 + OH- -> AlO2- + 2H2O}\\$
\end{enumerate}

\subparagraph{将 \ce{Al^3+}滴入 \ce{NaOH}溶液中}

\begin{enumerate}
	\item 先无明显现象:$\ce{Al^3+ + 4OH- -> AlO2- + H2O}\\$
	\item 后产生白色沉淀:$\ce{Al^3+ + 3AlO2- + 6H2O -> 4Al3(OH)3 v}\\$
\end{enumerate}

\paragraph{与氨水反应}

$\ce{Al^3+ + NH3*H2O -> Al(OH)3 v + 3NH4+}\\$

\paragraph{双水解反应}

\begin{itemize}
	\item $\ce{Al^3+ + 3HCO3- -> Al(OH)3 v + 3CO2 ^}$
	\item $\ce{Al^3+ + 3CO3^2- + 3H2O -> Al(OH)3 v + 3HCO3-}$
	\item $\ce{Al^3+ + 3AlO2- + 6H2O -> 4Al(OH)3 v}$
	\item $\ce{2Al^3+ + 3S^2- + 6H2O -> 2Al(OH)3 v + 3H2S ^}$
	\item $\ce{AlO2- + NH4+ + H2O -> 4Al(OH)3 v + NH3 ^}$
	\item $\ce{2Al^3+ + 3SiO3^2- + 6H2O -> 2Al(OH)3 v + 3H2SiO3 v}$
\end{itemize}

\subsection{偏铝酸根}

\paragraph{与强酸相互滴加}缓慢滴加并搅拌

\subparagraph{将 \ce{H2SO4}滴入 \ce{AlO2-}溶液中}

\begin{enumerate}
	\item 先出现白色沉淀:$\ce{AlO2- + H+ + H2O -> Al(OH)3 v}\\$
	\item 后沉淀消失:$\ce{Al(OH)3 + 3H+ -> Al^3+ + 3H2O}\\$
\end{enumerate}

\subparagraph{将 \ce{AlO2-}滴入 \ce{H2SO4}溶液中}

\begin{enumerate}
	\item 先无明显现象:$\ce{AlO2- + 4H+ -> Al^3+ + 2H2O}\\$
	\item 后产生白色沉淀:$\ce{Al^3+ + 3AlO2- + 6H2O -> 4Al3(OH)3 v}\\$
\end{enumerate}

\paragraph{与碳酸反应}

立即生成 \ce{Al(OH)3}沉淀且不溶解。

\begin{itemize}
	\item \ce{CO2}过量:$\ce{AlO2- + 2H2O + CO2 -> Al(OH)3 v + HCO3-}\\$
	\item \ce{CO2}少量:$\ce{2AlO2- + 3H2O + CO2 -> 2Al(OH)3 v + CO3^2-}\\$
\end{itemize}

\paragraph{与铵盐溶液反应}

$\ce{NH4+ + AlO2- + H2O -> Al(OH)3 v + NH3 ^}\\$


\subsection{氢氧化铝}

\subsubsection{物理性质}

\begin{itemize}
	\item 白色胶状沉淀
\end{itemize}

\subsubsection{制备}

\begin{itemize}
	\item $\ce{Al^3+ + NH3*H2O -> Al(OH)3 v + 3NH4+}\\$
	\item $\ce{AlO2- + 2H2O + CO2 -> Al(OH)3 v + HCO3-}\\$
	\item $\ce{Al^3+ + 3AlO2- + 6H2O -> 4Al3(OH)3 v}\\$
\end{itemize}


\subsection{总结}

\begin{figure}[h]
	 \centering\includegraphics[scale=0.8]{res/Al.pdf}
\end{figure}
	%\clearpage
\section{Fe}
\subsection{铁单质}
\subsubsection{物理性质}
\begin{itemize}
	\item 银白色固体,有金属性光泽;
	\item 容易被磁铁吸引;
	\item 地壳中居第四位;
\end{itemize}

\subsubsection{化学性质}
	铁元素性质活泼,有较强的还原性,主要化合价为+2价和+3价。
	\paragraph{与非金属单质反应} 
		\begin{itemize}
			\item $\ce{3Fe + 2O2 ->[\text{点燃}] Fe3O4}$
			\item $\ce{2Fe + 3Cl2 ->[\text{点燃}] FeCl3}$
			\item $\ce{Fe + S ->[\Delta] FeS}$
		\end{itemize}
	\paragraph{与水反应}
	铁在高温下与水蒸气反应
	$\ce{3Fe + 4H2O(g) ->[\text{高温}] Fe3O4 + 4H2}$
	\paragraph{与酸反应}
	铁遇到冷的浓硫酸或浓硝酸会钝化。
	\begin{itemize}
		\item 与非还原性酸:$\ce{Fe + 2H+ -> Fe^2+ + H2 ^}$
		\item 与还原性酸:$\ce{Fe + 4H+ + NO3- -> Fe^3+ + NO ^ + 2H2O}$
	\end{itemize}
	\paragraph{与盐溶液反应}
		\begin{itemize}
			\item 置换反应:$\ce{Fe + Cu^2+ -> Fe^2+ + Cu}$
			\item 与氯化铁溶液:$\ce{Fe + 2Fe^3+ -> 3Fe^2+}$ 
		\end{itemize}
		
\subsection{铁的氧化物}
\renewcommand\arraystretch{2}
\begin{center}
\begin{tabular}{|c|p{0.3\textwidth}<{\centering}|p{0.3\textwidth}<{\centering}|p{0.3\textwidth}<{\centering}|}
	\hline
	名称&氧化亚铁&氧化铁&四氧化三铁\\\hline
	俗称&-&铁红&磁性氧化铁\\\hline
	化学式&\ce{FeO}&\ce{Fe2O3}&\ce{Fe3O4}\\\hline
	化合价&+2&+3&+2、+3\\\hline
	物理性质&黑色粉末&\textcolor[rgb]{0.541,0.149,0.078}{红褐色}粉末&黑色晶体\\\hline
	与\ce{CO}反应&$\ce{FeO + CO ->[\Delta] Fe + CO2}$&$\ce{Fe2O3 + 3CO ->[\Delta] 2Fe + 3CO2}$&$\ce{Fe3O4 + 4CO ->[\Delta] 3Fe + 4CO2}$\\\hline
	与\ce{H2}反应&$\ce{FeO + H2 ->[\Delta] Fe + H2O}$&$\ce{Fe2O3 + 3H2 ->[\Delta] 2Fe + 3H2O}$&$\ce{Fe3O4 + 4H2 ->[\Delta] 3Fe + 4H2O}$\\\hline
	与酸反应&$\ce{FeO + 2H+ -> Fe^2+ + H2O}$&$\ce{Fe2O3 + 6H+ -> 2Fe^3+ + 3H2O}$&$\ce{Fe3O4 + 8H+ -> Fe^2+ + 2Fe^3+ + 4H2O}$\\\hline
\end{tabular}
\end{center}

\subsection{铁的水化物}
\subsubsection{比较\ce{Fe(OH)2}和\ce{Fe(OH)3}}
\begin{center}
\begin{tabular}{|c|c|c|}
	\hline
	名称&氢氧化亚铁&氢氧化铁\\\hline
	化学式&\ce{Fe(OH)2}&\ce{Fe(OH)3}\\\hline
	物理性质&白色固体&\textcolor[rgb]{0.541,0.149,0.078}{红褐色}固体\\\hline
	与酸反应&$\ce{Fe(OH)2 + 2H+ -> Fe^2+ + 2H2O}$&$\ce{Fe(OH)3 + 3H+ -> Fe^3+ + 3H2O}$\\\hline
	受热分解&$\ce{Fe(OH)2 ->[\Delta] FeO + H2O}$&$\ce{2Fe(OH)3 ->[\Delta] Fe2O3 + 3H2O}$\\\hline
	制备&$\ce{FeCl2 + 2NaOH -> Fe(OH)2 v + 2NaCl}$&$\ce{FeCl3 + 3NaOH -> Fe(OH)3 v + 3NaCl}$\\\hline
\end{tabular}
\end{center}
\subsubsection{\ce{Fe(OH)2}和\ce{Fe(OH)3}的转化}
\ce{Fe(OH)2}在空气中可以迅速被氧化成\ce{Fe(OH)3}。现象是由\textbf{白色絮状沉淀}迅速变成\textcolor[rgb]{0.231,0.301,0.219}{灰绿色},最后变成\textcolor[rgb]{0.541,0.149,0.078}{红褐色}。
$$\ce{4Fe(OH)2 + O2 + 2H2O -> 4Fe(OH)3}$$

\subsection{铁三角(铁、亚铁盐、铁盐)}
\begin{figure}[h]
\centering
\includegraphics[scale=0.8]{res/Fe.pdf}
\end{figure}

\paragraph{亚铁盐}
含有\ce{Fe^2+}的溶液呈\textcolor[rgb]{0.625,0.8,0.7}{浅绿色},\ce{Fe^2+}既有氧化性,又有还原性。
\paragraph{铁盐}
含有\ce{Fe^3+}的溶液呈\textcolor[rgb]{0.835,0.611,0.247}{棕黄色}, \ce{Fe^3+}具有氧化性。含有\ce{Fe^3+}的盐溶液遇到\ce{KSCN}溶液时变成红色。

	%\clearpage
\section{Si}
\subsection{硅单质}
\subsubsection{物理性质}
\begin{itemize}
	\item 分类:无定形硅、晶体硅(结构类似金刚石,原子晶体)
	\item 灰黑色晶状固体
	\item 质地较脆
	\item 半导体
\end{itemize}
\subsubsection{化学性质}
\paragraph{与非金属单质反应} 
	\begin{itemize}
		\item $\ce{Si + O2 ->[\text{高温}] SiO2}$
		\item $\ce{Si + 2Cl2 ->[\Delta] SiCl4}$
		\item $\ce{Si + 2F2 -> SiF4}$
		\item $\ce{Si + C ->[\text{高温}] \underset{\text{金刚砂}}{SiC}}$
	\end{itemize}
\paragraph{与水反应}
$\underbrace{\ce{Si + H2O + 2NaOH -> Na2SiO3 + 2H2 ^}}_{\text{野外制氢}}$
\paragraph{精炼}
\begin{enumerate}
	\item $\ce{Si + Cl2 ->[\Delta] SiCl4}$
	\item $\ce{SiCl4 + 2H2 ->[\text{高温}] 4HCl + Si}$
\end{enumerate}
	
\subsection{硅的氧化物}
最简式:\ce{SiO2}(分子晶体)
\subsubsection{物理性质}
\begin{itemize}
	\item 透明、硬度大、熔点高
\end{itemize}
\subsubsection{化学性质}
\paragraph{酸性氧化物}
\subparagraph{与强碱反应}
$\underbrace{\ce{SiO2 + 2NaOH -> Na2SiO3 + H2O}}_{\text{装NaOH溶液不用玻璃塞}}$
\subparagraph{与唯一一种酸氢氟酸反应}
$\underbrace{\ce{SiO2 + 4HF -> SiF4 ^ + 2 H2O}}_{\text{腐蚀玻璃、玻璃雕花}}$
(气标!气标!!)
\subparagraph{与碱性氧化物反应}
氧化硅与碱性氧化物反应,不与水反应(与水反应产物为硅酸,是沉淀,阻止反应进行)
\begin{itemize}
	\item $\ce{SiO2 + Na2O ->[\text{高温}] Na2SiO3}$
	\item $\ce{SiO2 + CaO ->[\text{高温}] CaSiO3}$
\end{itemize}
\subparagraph{与碱性盐反应}
\begin{itemize}
	\item $\underbrace{\ce{SiO2 + Na2CO3 ->[\text{高温}] Na2SiO3 + CO2 ^}}_{\text{制作玻璃}}$
	\item $\underbrace{\ce{SiO2 + CaCO3 ->[\text{高温}] CaSiO3 + CO2 ^}}_{\text{造渣反应}}$
\end{itemize}
\subparagraph{与碳反应}
\begin{itemize}
	\item $\ce{SiO2 + 2C ->[\text{高温}] Si + 2CO ^}$
	\item $\ce{SiO2 + 3C ->[\text{高温}] SiC + 3CO ^}$
\end{itemize}
\subparagraph{精炼}
\begin{enumerate}
	\item $\ce{SiO2 + 4Mg ->[\text{高温}] Mg2Si + 2MgO}$
	\item $\ce{Mg2Si + 4HCl -> 2MgCl2 + SiH4 ^}$
	\item $\ce{SiH4 + 2O2 -> SiO2 + 2H2O}$(自然)
\end{enumerate}


\subsection{硅的水化物(硅酸、原硅酸)}
硅酸:\ce{H2SiO3}、、
原硅酸:\ce{H4SiO4}
\subsubsection{物理性质}
白色胶状沉淀
\subsubsection{化学性质}
\paragraph{弱酸性}
不使酸碱指示剂变色
\subparagraph{硅酸电离}
$\left\{\begin{array}{lr}
	\ce{H2SiO3 <=> H+ + HSiO3-}\\
	\ce{H2SiO3- <=> H+ + SiO3^2-}\\
\end{array}\right.$
\subparagraph{原硅酸电离}
$\left\{\begin{array}{lr}
	\ce{H4SiO4 <=> H+ + H3SiO4-}\\
	\ce{H3SiO4- <=> H+ + H2SiO4^2-}\\
	\ce{H2SiO4- <=> H+ + HSiO4^3-}\\
	\ce{HSiO4- <=> H+ + SiO4^4-}\\
\end{array}\right.$
\paragraph{不稳定沉淀}
\begin{itemize}
	\item $\ce{H4SiO4 -> H2SiO3 + H2 ^}$
	\item $\ce{H2SiO3 ->[\Delta] SiO2 + H2O}$
	\item $\ce{H2SiO3 ->[\Delta] SiO2*xH2O + H2O}$
\end{itemize}
\paragraph{与强碱反应}
\subparagraph{与氢氧化钠反应}
$\ce{H2SiO3 + 2NaOH -> Na2SiO3 + 2H2O}$
\subparagraph{不与氨气反应}
$\ce{SiO3^2- + 2NH4+ -> H2SiO3 v + 2NH3 ^}$
\subsubsection{制备}
\paragraph{\ce{SiO2}无法一步变成\ce{H2SiO3}}
$\left\{\begin{array}{lr}
	\ce{SiO2 + 2NaOH -> Na2SiO3 + H2O}\\
	\ce{Na2SiO3 + 2HCl -> 2NaCl + H2SiO3 v}\\
\end{array}\right.$

\subsection{硅酸盐}
\subsubsection{物理性质}
\ce{K2SiO3}和\ce{Na2SiO3}溶于水,其余硅酸盐微溶于水。
\subsubsection{化学性质}
\begin{itemize}
	\item $\left\{\begin{array}{lr}
				\ce{Na2SiO3 + CO2 + H2O -> Na2CO3 + H2SiO3 v}\\
				\ce{Na2SiO3 + 2CO2 + 2H2O -> 2NaHCO3 + H2SiO3 v}\\
			\end{array}\right.$
	\item $\left\{\begin{array}{lr}
				\ce{Na2SiO3 + 6HF -> SiF4 ^ + 2NaF + 3H2O}\\
				\underbrace{\ce{CaSiO3 + 6HF -> SiF4 ^ + CaF2 + 3H2O}}_{\text{产物硅酸不稳定生成\ce{SiO2},继续与氢氟酸反应}}\\
			\end{array}\right.$
\end{itemize}
\subsubsection{硅酸盐的拆分}
$活泼金属氧化物\longrightarrow 较活泼金属氧化物\longrightarrow 二氧化硅\longrightarrow 水$
\begin{itemize}
	\item \ce{Na2SiO3}:\ce{Na2O*SiO2}
	\item \ce{CaSiO3}:\ce{CaO*SiO2}
	\item \ce{Al2(Si2O5)(OH)4)}:\ce{Al2O3*2SiO2*2H2O}
\end{itemize}

\subsection{用途与俗称}
\subsubsection{用途}
\begin{itemize}
	\item \ce{Si}(不透明):硅芯片、太阳能电池板
	\item \ce{SiO2}(透明):玻璃、石英玻璃、硅胶(\ce{mSiO2*nH2O},干燥剂)、光导纤维
	\item \ce{SiO3^2-}盐:水泥、陶瓷、防火材料等无机非金属材料
	\item \ce{H2SiO3}:制硅胶
\end{itemize}
\subsubsection{俗称}
\begin{itemize}
	\item \ce{SiO2}:水晶、玛瑙、石英
	\item \ce{Na2SiO3}水溶液:水玻璃
	\item \ce{Na2SiO3}:泡花碱
\end{itemize}
	%\clearpage
\section{Cl}
\subsection*{氯相关}
\subsubsection{含氯酸}
从上至下,酸性递增,氧化性递减。
\begin{itemize}
	\item \ce{HClO}:次氯酸
	\item \ce{HClO2}:亚氯酸
	\item \ce{HClO3}:氯酸
	\item \ce{HClO4}:高氯酸
\end{itemize}
\subsubsection{卤素}
\begin{itemize}
	\item \ce{F}:无正价,氧化性最强的单质
	\item \ce{Cl}:黄绿色气体
	\item \ce{Br}:常温下唯一液态非金属单质,保存液溴需水封,海水元素
	\item \ce{I}:易升华
\end{itemize}
\begin{itemize}
	\item \ce{AgF}:可溶于水
	\item \ce{AgCl}:白色沉淀
	\item \ce{AgBr}:淡黄色沉淀
	\item \ce{AgI}:黄色沉淀,用于人工降雨
\end{itemize}
\paragraph{海水提溴}
$$
\ce{{海水} ->[{晒盐}] {盐卤} ->[{通入}Cl2] Br2(aq) ->[{吹热空气或水蒸气}] Br2(g) ->[{热饱和纯碱}] {溴酸盐和溴盐溶液} ->[{稀硫酸酸化}] {Br2}}
$$
$\left\{\begin{array}{lr}
	\ce{Cl2 + 2Br- -> Br2 + 2Cl-}\\
	\ce{3Br2 + 3CO3^2- -> 5Br- + BrO3- + 3CO2}\\
	\ce{5Br- + BrO3- + 6H+ -> 3Br2 + 3H2O}\\
\end{array}\right.$
\paragraph{海带提碘}
$$
\ce{{海带} -> {烧碱灰} ->[{泡水浸取}] ->[Cl2] I2}
$$
\paragraph{拟卤素}
$\underset{\text{氰}}{\ce{CN}}$、$\underset{\text{硫氰}}{\ce{SCN}}$、$\underset{\text{氧氰}}{\ce{OCN}}$

\subsection{盐酸}
\subsubsection{物理性质}
无色、有刺激性气味液体。
\subsubsection{化学性质}
\paragraph{酸性}
产物中有盐
\begin{itemize}
	\item $\ce{2H+ + Fe -> Fe^2+ + H2 ^}$
	\item $\ce{H+ + OH- -> H2O}$
	\item $\ce{2H+ + CaO -> Ca2+ + H2O}$
	\item $\ce{2H+ + CO3^2- -> CO2 ^ + H2O}$
\end{itemize}
\paragraph{氧化性}
盐酸的氧化性由\ce{H+}体现
\begin{itemize}
	\item $\ce{2H+ + Fe -> Fe^2+ + H2 ^}$
\end{itemize}
\paragraph{还原性}
\begin{itemize}
	\item $\underbrace{\ce{4HCl({浓}) + MnO2 ->[\Delta] MnCl2 + CL2 ^ + 2H2O}}_{\text{实验室制氯气}}$
	\item $\left\{\begin{array}{lr}
			\ce{16HCl + 2KMnO4 -> 2KCl + 5Cl2 ^ + 2MnCl2 + 8H2O}\\
			\ce{14HCl + K2Cr2O7 -> 2KCl + 3Cl2 ^ + 2CrCl3 + 7H2O}\\
			\ce{6HCl + KClO3 -> KCl + 3Cl2 ^ + 3H2O}\\
			\ce{14HCl + PbO2 -> PbCl2 + Cl2 ^ + 2H2O}\\
			\ce{6HCl + NaBiO3 -> NaCl + Cl2 ^ + BiCl2 + 3H2O}\\
		\end{array}\right.$
\end{itemize}
\subsubsection{制备}
\paragraph{工业}
\begin{enumerate}
	\item $\ce{2NaCl + 2H2O ->[{通电}] 2NaOH + H2 ^ + Cl2 ^}$
	\item $\ce{H2 + Cl2 ->[{点燃}] 2HCl}$
\end{enumerate}
\paragraph{实验室}
\begin{itemize}
	\item $\ce{NaCl + H2SO4({浓}) ->[\Delta] NaHSO4 + HCl ^}$
	\item $\ce{2NaCl + H2SO4({浓}) ->[\Delta] Na2SO4 + 2HCl ^}$
\end{itemize}
\subsection{氯气}
\subsubsection{物理性质}
\begin{itemize}
	\item \textcolor[rgb]{0.745,0.752,0.317}{黄绿色}气体
	\item 密度大于空气,加压易液化
	\item 难溶于饱和食盐水,可溶于水,易溶于\ce{CCl4}。
\end{itemize}
\subsubsection{化学性质}
\paragraph{助燃性}
强氧化性
\begin{itemize}
	\item $\ce{H2 + Cl2 ->[{点燃}] 2HCl}$(苍白色火焰)
	\item $\ce{2Fe + 3Cl2 ->[{点燃}] 2FeCl3}$(产物是三价铁)
	\item $\ce{Cu + Cl2 ->[{点燃}] CuCl2}$
	\item $\ce{2Na + Cl2 ->[{点燃}] 2NaCl}({白烟黄光})$
	\item 磷在氯气中燃烧产生白色烟雾$\left\{\begin{array}{lr}
			\ce{2P + 5Cl2 ->[{点燃}] 2PCl5}({烟})\\
			\ce{2P + 3Cl2 ->[{点燃}] 2PCl3}({雾})\\
		\end{array}\right.$
	\item $\left\{\begin{array}{lr}
			\ce{PCl3 + 3H2O -> H3PO3 + 3HCl}\\
			\ce{PCl5 + 4H2O -> H3PO4 + 5HCl}\\
		\end{array}\right.$
\end{itemize}
\paragraph{氧化性和还原性}
\begin{itemize}
	\item $\left\{\begin{array}{lr}
			\ce{H2O + Cl2 <=> HCl + HClO}\\
			\ce{H2O + Cl2 <=> H+ + Cl- + HClO}\\
		\end{array}\right.$
	\item $\left\{\begin{array}{lr}
			\ce{NaOH + Cl2 -> NaCl + \underset{\text{84消毒液、漂白粉}}{NaClO} + H2O}\\
			\ce{2Ca(OH)2 + 2Cl2 -> CaCl2 + \underset{\text{漂白精、漂白粉}}{Ca(ClO)2} + 2H2O}\\
		\end{array}\right.$
	\item $\left\{\begin{array}{lr}
			\ce{6NaOH + 3Cl2 ->[\Delta] 5NaCl + NaClO3 + 3H2O}\\
			\ce{6KOH + 3Cl2 ->[\Delta] 5KCl + KClO3 + 3H2O}\\
		\end{array}\right.$
	\item $\ce{2H2O + Cl2 + SO2 -> HCl + H2SO4}$
\end{itemize}
\subsubsection{制备}
\paragraph{工业}
\begin{itemize}
	\item $\ce{2NaCl + 2H2O ->[{通电}] 2NaOH + H2 ^ + Cl2 ^}$
	\item $\ce{2NaCl(l) ->[{通电}] 2Na + Cl2 ^}$
\end{itemize}
\paragraph{实验室}
\begin{itemize}
	\item $\ce{MnO2 + 4HCl({浓}) ->[\Delta] Cl2 ^ + MnCl2 + 2H2O}$
\end{itemize}
\subsubsection{除杂}
\begin{itemize}
	\item \ce{Cl2}(\ce{HCl}):饱和食盐水(溶液度:\ce{HCl} > \ce{NaCl} > \ce{Cl2})
	\item \ce{HCl}(\ce{Cl2}):\ce{CCl4}
	\item \ce{CO2}(\ce{HCl}):饱和\ce{NaHCO3}溶液
\end{itemize}
\subsubsection{氯水}
\paragraph{成分}
\begin{itemize}
	\item 分子:\ce{H2O}、\ce{Cl2}、\ce{HClO}
	\item 离子:\ce{Cl-}、\ce{H+}、\ce{ClO-}、\ce{OH-}
\end{itemize}
\paragraph{检验}
\begin{itemize}
	\item \ce{Cl2}:\ce{FeCl2}溶液由\textcolor[rgb]{0.625,0.8,0.7}{浅绿色}变为\textcolor[rgb]{0.835,0.611,0.247}{棕黄色}
	\item \ce{Cl-}:加入硝酸酸化的\ce{AgNO3}溶液,产生白色沉淀
	\item \ce{HClO}:有色布条褪色
	\item \ce{H+}:pH试纸先变红,再褪色
\end{itemize}
\subsubsection{鉴别}
湿润淀粉碘化钾试纸变为\textcolor[rgb]{0.556,0.827,0.898}{蓝色}
$\ce{Cl2 + 2KI -> 2KCl + I2}$
\subsection{次氯酸}
\paragraph{化学式}
\ce{HClO}
\subsubsection{化学性质}
\paragraph{见光分解}
$\ce{2HClO ->[{光}] 2HCl + O2 ^}$
\paragraph{酸性}
\ce{H2CO3} > \ce{HClO} > \ce{HCO3-}
\paragraph{氧化性}
$\ce{HClO + SO2 + H2O -> HCl + H2SO4}$
\subsection{含氯酸盐}
\subsubsection{\ce{NaClO}}
\paragraph{次氯酸钠的变质}
$\left\{\begin{array}{lr}
	\ce{CO2 + NaClO + H2O -> HClO + NaHCO3}\\
	\ce{2HClO ->[{光}] 2HCl + O2 ^}\\
\end{array}\right.$
\paragraph{\ce{SO2}通入\ce{NaClO3}溶液}
$\ce{ClO- + SO2 + H2O -> Cl- + 2H+ + SO4^2-}$
\subsubsection{\ce{Ca(ClO)2}}
\paragraph{次氯酸钙的变质}
$\left\{\begin{array}{lr}
	\ce{CO2 + Ca(ClO)2 + H2O -> 2HClO + CaCO3 v}\\
	\ce{2HClO ->[{光}] 2HCl + O2 ^}\\
\end{array}\right.$
\paragraph{\ce{SO2}通入\ce{Ca(ClO3)2}溶液}
$\ce{Ca^2+ + ClO- + SO2 + H2O -> Cl- + 2H+ + CaSO4 v}$
\subsubsection{\ce{Cl2}逐渐通入\ce{FeI2}和\ce{FeBr2}混合溶液}
\begin{enumerate}
	\item $\ce{Cl2 + 2I- -> 2Cl- + I2}$
	\item $\ce{Cl2 + 2Fe^2+ -> 2Cl- + 2Fe^3+}$
	\item $\ce{Cl2 + 2Br- -> 2Cl- + Br2 ^}$
	\item $\ce{5Cl2 + 6H2O + I2 -> 12H+ + 2IO3- + 10Cl-}$
\end{enumerate}
\subsubsection{\ce{Cl2}逐渐通入\ce{Na2CO}溶液}
\begin{equation}\label{equ:E1}
		\ce{H2O + Cl2 <=> HCl + HClO}
\end{equation}
\begin{equation}\label{equ:E2}
		\ce{HCl + Na2CO3 -> NaCl + NaHCO3}
\end{equation}
\begin{equation}\label{equ:E3}
		\ce{HCl + NaHCO3 -> NaCl + H2O + CO2 ^}
\end{equation}
\begin{equation}\label{equ:E4}
		\ce{HClO + Na2Co3 <=> NaClO + NaHCO3}
\end{equation}
注意\ce{HClO}和\ce{NaHCO3}不反应。
\begin{enumerate}
	\item $\ce{2Na2CO3 + Cl + H2O -> 2NaHCO3 + NaCl + NaClO}$(\ref{equ:E1}+\ref{equ:E2}+\ref{equ:E4})
	\item $\ce{Cl2 + Na2CO3 + H2O -> NaCl + NaHCO3 + HClO}$(\ref{equ:E1}+\ref{equ:E2})
	\item $\ce{Na2CO3 + 2Cl2 + H2O -> CO2 ^ + 2NaCl + 2HClO}$(\ref{equ:E1}+\ref{equ:E2}+\ref{equ:E3})
\end{enumerate}
	%\clearpage
\section{S}

\subsection{硫化氢}
\subsubsection{物理性质}
\begin{itemize}
	\item 无色、有刺激性气味(臭鸡蛋味)、有毒气体
	\item 能溶于水
	\item \ce{H2S}水溶液俗称氢硫酸,有毒
	\begin{itemize}
	\item 碘酸:\ce{HIO}
	\item 碘化氢:\ce{HI}
	\item 氢碘酸:\ce{HI}水溶液
	\end{itemize}
\end{itemize}
\subsubsection{化学性质}
\paragraph{弱酸性}
\subparagraph{与碱生成对应酸式/正盐}
\subparagraph{与一些盐反应}
\begin{itemize}
	\item $\ce{H2S + CuSO4 -> CuS v + H2SO4}$(强酸置弱酸)
	\item $\ce{PbAc2 + H2S -> PbS v + 2HAc}$(鉴别硫化氢:$\underset{\text{醋酸铅}}{\ce{PbAc2}}$试纸变黑)
\end{itemize}
\paragraph{不稳定性}
高温易分解
\paragraph{可燃性}
\begin{itemize}
	\item $\ce{2H2S + 3O2 ->[\text{点燃}] 2SO2 + 2H2O}$
	\item $\ce{2H2S + O2 ->[\text{点燃}] 2S + 2H2O}$
\end{itemize}
\paragraph{强还原性}
\begin{itemize}
	\item $\ce{2H2S + SO2 -> 3S v + 2H2O}$
	\item $\ce{2H2S(aq) + O2 -> 2S v + 2H2O}$
	\item $\ce{H2S + X2 -> 2HX + S v}$
	\item $\left\{\begin{array}{lr}
			\ce{H2S + H2O2 -> 2H2O + S v}\\
			\ce{H2S + 4H2O2 -> H2SO4 + 4H2O}\\
		\end{array}\right.$
\end{itemize}
\subsubsection{制备}
向上双管排气法收集
除杂:\ce{NaOH}
\begin{itemize}
	\item $\ce{FeS + H2SO4 -> H2S ^ + FeSO4}$
	\item $\ce{ZnS + H2SO4 -> H2S ^ + ZnSO4}$
\end{itemize}

\subsection{硫单质}
\subsubsection{物理性质}
\begin{itemize}
	\item \textcolor[rgb]{0.905,0.803,0.376}{黄色}硫固体/\textcolor[rgb]{0.874,0.890,0.756}{淡黄色}硫粉/白色纳米尺度的硫
	\item 难溶于水、微溶于酒精、易溶于二硫化碳
	\item 熔沸点低,存在多种同素异形体
\end{itemize}
\subsubsection{化学性质}
\paragraph{与金属反应}
主要生成低价化合物
\begin{itemize}
	\item $\ce{S + Fe ->[\Delta] FeS}$
	\item $\ce{S + 2Cu ->[\Delta] \underset{\text{硫化亚铜}}{Cu2S}}$
	\item $\underbrace{\ce{3S + 2Al ->[\Delta] Al2S3}}_{\text{	高中唯一制\ce{Al2S3}的方法}}$
	\item $\underbrace{\ce{S + Hg -> HgS}}_{\text{除汞}}$
\end{itemize}
\paragraph{与非金属反应}
\begin{itemize}
	\item $\ce{S + 3F2 -> \underset{\text{变压器涂层}}{SF6}}$
	\item $\ce{S + O2 ->[\Delta\text{或点燃}] SO2}$
	\item $\ce{S + H2 <=>[\text{高温}] H2S}$
\end{itemize}
\paragraph{还原性}
\begin{itemize}
	\item $\left\{\begin{array}{lr}
			\ce{S + 4HNO3(\text{浓}) ->[\Delta] SO2 ^ + 4NO2 ^ + 2H2O}\\
			\ce{S + 2H2SO4(\text{浓}) ->[\Delta] 3SO2 ^ + 2H2O}\\
		\end{array}\right.$
	\item $\ce{S + 3H2O2 -> H2SO4 + 2H2O}$
\end{itemize}
\paragraph{除硫粉}
\begin{enumerate}
	\item 用\ce{CS2}洗涤
	\item 用热的氢氧化钠溶液洗涤:$\ce{3S + 6NaOH ->[\Delta] 2Na2S + Na2So3 + 3H2O}$
\end{enumerate}
\paragraph{歧化和归中}
\begin{itemize}
	\item 硫单质:$\left\{\begin{array}{lr}
			\underbrace{\ce{S + OH- ->[\Delta] S^2- + SO3^2- + H2O}}_{\text{碱性歧化}}\\
			\underbrace{\ce{S^2- + SO3^2- + H+ -> S v + H2O}}_{\text{酸性归中}}\\
		\end{array}\right.$
	\item 卤素加热: $\left\{\begin{array}{lr}
			\ce{X2 + OH- ->[\Delta] X- + XO3- + H2O}\\
			\ce{X- + XO3- + H+ -> X2 ^ + H2O}\\
		\end{array}\right.$
	\item 卤素不加热:$\left\{\begin{array}{lr}
			\ce{X2 + OH- -> X- + XO- + H2O}\\
			\ce{X- + XO- + H+ -> X2 ^ + H2O}\\
		\end{array}\right.$
\end{itemize}
\subsection{二氧化硫}
\subsubsection{物理性质}
\begin{itemize}
	\item 无色、刺激性气味、有毒气体
	\item 易溶于水
\end{itemize}
\subsubsection{化学性质}
\paragraph{酸性}
\begin{itemize}
	\item 与强碱:$\left\{\begin{array}{lr}
					\ce{SO2 + 2NaOH -> Na2SO3 + H2O}\\
					\ce{SO2 + NaOH -> NaHSO3}\\
					\ce{SO2 + Ca(OH)2 -> CaSO3 v + H2O}\\
				\end{array}\right.$
	\item 与碱性氧化物:$\left\{\begin{array}{lr}
					\underbrace{\ce{SO2 + CaO -> CaSO3}}_{\text{蜂窝煤脱硫}}\\
					\ce{2CaSO3 + O2 ->[\Delta] 2CaSO4}\\
				\end{array}\right.$
	\item 与水反应:$\left\{\begin{array}{lr}
					\ce{SO2 + H2O <=> H2SO3}\\
					\ce{H2SO3 <=> H+ + HSO3-}\\
					\ce{HSO3- <=> H+ + SO3^2-}\\
				\end{array}\right.$
	\item 酸性比盐酸弱:不与\ce{BaCl2}溶液反应生成沉淀
	\item 与\ce{BaCl2}和\ce{NH3*H2O}溶液:$\left\{\begin{array}{lr}
			\ce{SO2 + 2NH3*H2O -> (NH4)2SO3 + H2O}\\
			\ce{(NH4)2SO3 + BaCl2 -> BaSO3 v + 2NH4Cl}\\
		\end{array}\right.$
	\item 与\ce{BaCl2}和\ce{Cl2}溶液:$\left\{\begin{array}{lr}
					\ce{SO2 + 2H2O + Cl2 -> H2SO4 + 2HCl}\\
					\ce{H2SO4 + BaCl2 -> 2HCl + BaSO4 v}\\
				\end{array}\right.$
\end{itemize}
\paragraph{氧化性}
$\ce{2H2S + SO2 -> 3S v + H2O}$(仅此一个反应能体现氧化性)
\begin{itemize}
	\item \ce{SO2}通入\ce{Na2S}溶液:$\left\{\begin{array}{lr}
			\ce{SO2 + H2O -> H2SO3}\\
			\ce{H2SO3 + Na2S -> Na2SO3 + H2S ^}\\
			\ce{2H2S + SO2 -> 3S v + H2O}\\
		\end{array}\right.$
\end{itemize}
\paragraph{还原性}
\begin{itemize}
	\item $\ce{SO2 + H2O2 -> H2SO4}$
	\item $\ce{SO2 + Na2O2 -> Na2SO4}$
	\item $\ce{SO2 + 2Fe^3+ + 2H2O -> 2Fe^2+ + SO4^2- + 4H+}$
	\item $\ce{5SO2 + 2MnO4- + 2H2O -> 2Mn^2+ + 5SO4^2- + 4H+}$
	\item $\ce{SO2 + HClO + H2O -> 3H+ + Cl- + SO4^2-}$
	\item $\ce{NO2 + SO2 -> NO + SO3}$
	\item $\left\{\begin{array}{lr}
			\ce{SO2 + 2H2O + X2 -> H2SO4 + 2HX}\\
			\ce{SO2 + 2H2O + Cl2 -> H2SO4 + 2HCl}\\
		\end{array}\right.$
\end{itemize}
\paragraph{漂白性}
\ce{SO2}使\textcolor[rgb]{0.721,0.207,0.105}{品红溶液}褪色,加热后红色复现。原理:与特定有机染料结合,生成无色或浅色物质;加热可逆
\begin{itemize}
	\item \ce{SO2}通入酸性高锰酸钾溶液褪色:还原性
	\item \ce{SO2}通入品红溶液褪色:漂白性
\end{itemize}
\subsubsection{硫酸型酸雨}
\begin{itemize}
	\item $\ce{SO2 -> SO3 -> H2SO4}$
	\item $\ce{SO2 -> H2SO3 -> H2SO4}$
\end{itemize}
\subsubsection{除杂}
\begin{itemize}
	\item \ce{SO2}(\ce{CO2}):\ce{NaHSO3}溶液
	\item \ce{CO2}(\ce{SO2}):\ce{NaHCO3}溶液或酸性高锰酸钾溶液
	\item \ce{SO2}(\ce{HCl}):\ce{NaHSO3}溶液
	\item \ce{SO2}(\ce{Cl2}):无法分开
\end{itemize}
\subsubsection{制备}
\begin{itemize}
	\item $\left\{\begin{array}{lr}
			\ce{Na2SO3 + H2SO4({浓}) -> Na2SO4 + CO2 ^ + H2O}\\
			\ce{Na2SO3 + H2SO4 ->[\Delta] Na2SO4 + CO2 ^ + H2O}\\
		\end{array}\right.$
	\item 装置:固液加热,含沸石
	\item 除杂(水):浓硫酸或无水氯化钙
	\item 收集:向上排空气(易溶于水,不能用排水法)
	\item 验满:湿润的蓝色石蕊试纸(酸性)或品红试纸(漂白性)
	\item 尾气处理:氢氧化钠溶液、放倒吸(工业用氨水,产物可做化肥)
\end{itemize}

\subsection{三氧化硫}
\subsubsection{物理性质}
\begin{itemize}
	\item 无色
	\item 常温液体、标况固体
	\item 溶于浓硫酸
\end{itemize}
\subsubsection{化学性质}
酸性氧化物,与水反应生成硫酸,放热。
$$
\ce{SO3 + CaO -> CaSO4}\\
\ce{SO3 + 2NaOH -> Na2SO4 + H2O}
$$
\subsubsection{除杂}
弱酸气体混有强酸气体杂质时,用弱酸的酸式盐溶液除杂。也可以利用杂质的氧化性或还原性除杂。
\begin{itemize}
	\item \ce{CO2}(\ce{SO2}):酸性高锰酸钾溶液、\ce{Fe2(SO4)3}溶液、\ce{NaHCO3}溶液
	\item \ce{H2S}(\ce{HCl}):饱和\ce{NaHS}溶液
	\item \ce{CO2}(\ce{H2S}):酸性高锰酸钾溶液、\ce{Fe2(SO4)3}溶液、\ce{CuSO4}溶液
\end{itemize}

\subsection{亚硫酸}
\subsubsection{化学性质}
\paragraph{不稳定性}
\begin{itemize}
	\item $\ce{H2SO3 ->[\Delta] H2O + SO2 ^}$
\end{itemize}
\paragraph{还原性}
\begin{itemize}
	\item $\ce{2H2SO3 + O2 <=> H2SO4}$
	\item $\ce{H2SO3 + Cl2 + H2O <=> H2SO4 + 2HCl}$
	\item $\ce{H2SO3 + H2O2 <=> H2SO4 + H2O}$
\end{itemize}
\paragraph{酸性}
亚硫酸是中强酸
\begin{itemize}
	\item \ce{NaHCO3}:显碱性
	\item \ce{NaHSO3}:显酸性
\end{itemize}

\subsection{硫酸}
\subsubsection{物理性质}
\begin{itemize}
	\item 无色粘稠状液体、不易挥发
	\item 吸水性
	\item 溶于水放热
\end{itemize}
\subsubsection{化学性质}
\paragraph{酸性}
\paragraph{脱水性(注意区分吸水性)}
酸性干燥剂
\paragraph{强氧化性}
\begin{itemize}
	\item 与金属反应:可与金属活动顺序表中铜及之前的物质反应,常温下使铁、铝钝化。
	\begin{itemize}
		\item $\ce{Cu + 2H2SO4(\text{浓}) ->[\Delta] CuSO4 + SO2 ^ + 2H2O}$
	\end{itemize}
	\item 与非金属反应:$\left\{\begin{array}{lr}
			\ce{C + 2H2SO4(\text{浓}) ->[\Delta] CuSO4 + SO2 ^ + 2H2O}\\
			\ce{S + 2H2SO4(\text{浓}) ->[\Delta] 3SO2 ^ + 2H2O}\\
		\end{array}\right.$
	\item 与化合物反应:$\left\{\begin{array}{lr}
			\ce{2Br- + SO4^2- + 4H+ -> Br2 + SO2 ^ + 2H2O}\\
			\ce{2Fe^2+ + SO4^2- + 4H+ -> 2Fe^3+ + SO2 ^ + H2O}\\
		\end{array}\right.$
\end{itemize}
\subsubsection{制备}
\paragraph{工业}
\subparagraph{沸腾炉}
煅烧黄铁矿
$$\ce{4FeS2 + 11O2 ->[\Delta] 2Fe2O3 + 8SO2}$$
\subparagraph{接触室}
\ce{V2O5}附着于网上
$$\ce{2SO2 + O2 ->[{催化剂}][\Delta] 2SO3}$$
\subparagraph{吸收塔}
$$\ce{SO3 + H2O -> H2SO4}$$
实际用浓硫酸吸收
$\left\{\begin{array}{lr}
	\ce{H2SO4 + SO3 -> \underset{\text{焦硫酸}}{\ce{H2S2O7}}}\\
	\ce{H2S2O7 + H2O -> 2H2SO4}\\
\end{array}\right.$
\subsection{含硫酸盐}
\subsubsection{\ce{FeSO4}}
$$
\ce{FeSO4 ->[\Delta] Fe2O3 + SO2 ^ + SO3 ^}
$$
\subsubsection{\ce{CuSO4}}
$$\left\{\begin{array}{lr}
	\ce{CuSO4 ->[\Delta] CuO + SO2}\\
	\ce{CuSO4 ->[\Delta\text{(更高温度)}] CuO + SO2 ^ + SO3 ^ + O2 ^}\\
\end{array}\right.$$
\paragraph{制备}
$\left\{\begin{array}{lr}
	\ce{2Cu + 2H2SO4(\text{稀}) + O2 ->[\Delta] 2CuSO4 + 2H2O}\\
	\ce{Cu + H2SO4(\text{稀}) + H2O2 -> CuSO4 + 2H2O}\\
\end{array}\right.$
\subsubsection{\ce{Na2S2O3}}
\begin{itemize}
	\item 无法在酸性条件下存在:$\ce{Na2S2O3 + 2HCl -> 2NaCl + H2O + SO2 ^ + S v}$
	\item 生成:$\ce{Na2So3 + S -> Na2S2O3}$
	\item 除氯剂:$\ce{Na2S2O3 + 4Cl2 + 10NaOH -> 8NaCl + 2Na2So4 + 5H2O}$
	\item 测定空气中\ce{I2}含量:$\ce{2Na2S2O3 + I2 -> B=Na2S4O6 + 2NaI}$
\end{itemize}

	%
\clearpage
\section{N}
\subsection{氨气}
\subsubsection{物理性质}
\begin{itemize}
	\item 无色、刺激性气体
	\item 极易溶于水
	\item 加压易液化(制冷剂)
\end{itemize}
\subsubsection{尾气处理防倒吸}
\ce{NH3}或\ce{HCl}等气体极易溶于水,直接通入水中会使水倒吸。在水层下放\ce{CCl4}层并将气体通入,可以防止倒吸。(\ce{NH3}和\ce{HCl}不溶于\ce{CCl4})
\subsubsection{喷泉实验}
\begin{center}
\begin{tabular}{|c|c|}
	\hline
	气体&液体\\\hline
	\ce{NH3}&水或稀\ce{H2SO4}\\\hline
	\ce{HCl}&水或\ce{NaOH}溶液\\\hline
	\ce{Cl2}&~\\\cline{1-1}
	\ce{CO2}&\ce{NaOH}\\\cline{1-1}
	\ce{SO2}&溶液\\\cline{1-1}
	\ce{H2S}&~\\\hline
\end{tabular}
\end{center}
\subsubsection{化学性质}
\paragraph{可燃性}
\begin{itemize}
	\item $\ce{4NH3 + 3O2 ->[\Delta{或点燃}] 2N2 + 6H2O}$
\end{itemize}
\paragraph{碱性}
唯一的碱性气体
\begin{itemize}
	\item $\ce{NH3 + HCl -> \underset{\text{白烟}}{\ce{NH4Cl}}}$
	\item $\ce{NH3 + H2O <=> NH3*H2O <=> NH4+ + OH-}$
\end{itemize}
\paragraph{还原性}
\begin{itemize}
	\item 催化氧化:$\ce{4NH3 + 5O2 <=>[Pt][\Delta] 4NO + 6H2O}$
	\item $\left\{\begin{array}{lr}
			\ce{2NH3 + 3Cl2 -> N2 + 6HCl}\\
			\ce{8NH3 + 3Cl2 -> N2 + 6\underset{\text{白烟:检验氯气泄漏}}{\ce{NH4Cl}}}\\
		\end{array}\right.$
	\item $\ce{2NH3 + CuO ->[\Delta] 3Cu + N2 + 3H2O}$
\end{itemize}
\subsubsection{检验与验满}
\begin{itemize}
	\item 检验:\ce{NH3}能使湿润红色石蕊试纸变蓝(没有紫色石蕊试纸)。
	\item 验满:沾取少量浓盐酸,置于瓶口,出现白烟。
\end{itemize}

\subsubsection{制备}
\begin{itemize}
	\item $\ce{Ca(OH)2 + 2NH4Cl ->[\Delta] CaCl2 + 2NH3 ^ + 2H2O}$
\end{itemize}
\subsubsection{用途}
制硝酸、氮肥、制冷剂

\subsection{氮气}
\subsubsection{物理性质}
\begin{itemize}
	\item 无色无味气体、难溶于水
\end{itemize}
\subsubsection{化学性质}
氮气常温下不活泼(氮氮三键)、高温下(氮原子)活泼。
\begin{itemize}
	\item $\ce{N2 + 3Mg ->[{点燃}] \underset{\text{\textcolor[rgb]{0.874,0.890,0.756}{淡黄色}}}{Mg3N2}}$
	\item $\ce{N2 + 3H2 <=>[{高温、高压}][{催化剂}] 2NH3}$
	\item $\ce{N2 + O2 ->[{高温}] 2NO}$
\end{itemize}
\subsubsection{制备}
\begin{itemize}
	\item $\ce{NaNO2 + NH4Cl ->[\Delta] NaCl + N2 ^ + 2H2O}$
\end{itemize}

\subsection{氮的氧化物}
\subsubsection{物理性质}
\begin{itemize}
	\item \ce{NO}:无色气体、有毒、难溶于水
	\item \ce{NO2}:\textcolor[rgb]{0.827,0.286,0.184}{红棕色}气体、有毒、与水反应
	\item \ce{N2O4}:无色气体、有毒、与水反应、化学性质类似\ce{NO2}、标况非气体
\end{itemize}
\subsubsection{化学性质}
\paragraph{一些实际发生的反应}
\begin{itemize}
	\item $\ce{2NO + O2 -> NO2}$(迅速转变为\textcolor[rgb]{0.827,0.286,0.184}{红棕色})
	\item $\ce{3NO2 + H2O -> 2HNO3 + NO}$(歧化)
	\item $\ce{2NO2 <=> N2O4}$
	\item $\ce{2NO2 + 2NaOH -> NaNO2 + NaNO3 + H2O}$
	\item $\ce{NO + NO2 + 2NaOH -> 2NaNO3 + H2O}$
\end{itemize}
\paragraph{推导反应(只能用于计算)}
\begin{itemize}
	\item $\ce{3NO2 + H2O -> 2HNO3 + NO}$
	\item $\ce{4NO2 + O2 + 2H2O -> 4HNO3}$
	\item $\ce{4NO + 3O2 + 2H2O -> 4HNO3}$
\end{itemize}
\paragraph{与氮的氢化物反应}
\begin{itemize}
	\item $\ce{6NO + 4NH3 ->[\Delta] 5N2 + 6H2O}$
	\item $\ce{6NO2 + 8NH3 ->[\Delta] 7N2 + 12H2O}$
	\item $\underbrace{\ce{N2O4 + 3N2H4 ->[\Delta] 3N2 + 4H2O}}_{\text{火箭推进}}$
\end{itemize}
\subsubsection{酸酐}
将可电离的\ce{H+}配合\ce{O}分解。
$$
\ce{H2SO4 -> SO3 + H2O}\\
\ce{2HNO3 -> N2O5 + H2O}\\
$$
\paragraph{化学性质}
与碱反应生成盐和水
\paragraph{与酸性氧化物的关系}
酸酐是酸性氧化物或非氧化物,酸性氧化物一定是酸酐。

\subsection{硝酸}
	\subsubsection{物理性质}
\begin{itemize}
	\item 无色、有刺激性气味
\end{itemize}
\subsubsection{化学性质}
\paragraph{氧化性}
活泼金属与硝酸反应时不生成氢气。
\begin{itemize}
	\item $\left\{\begin{array}{lr}
			\ce{Cu + 4HNO3({浓}) -> Cu(NO3)2 + 2NO2 ^ + 2H2O}\\
			\ce{Cu + 8HNO3({稀}) -> 3Cu(NO3)2 + 2NO ^ + 4H2O}\\
		\end{array}\right.$
	\item $\left\{\begin{array}{lr}
			\ce{Zn + 4HNO3({浓}) -> Zn(NO3)2 + 2NO2 ^ + 2H2O}\\
			\ce{Zn + 8HNO3({稀}) -> 3Zn(NO3)2 + 2NO ^ + 4H2O}\\
			\ce{4Zn + 10HNO3({更稀}) -> 4Zn(NO3)2 + N2O ^ + 5H2O}\\
			\ce{4Zn + 10HNO3({极稀}) -> 4Zn(NO3)2 + NH4NO3 + 3H2O}\\
		\end{array}\right.$
	\item $\ce{C + 4HNO3({浓})  ->[\Delta] 4NO2 ^ + CO2 ^ + 2H2O}$
\end{itemize}
\paragraph{不稳定性}
\begin{itemize}
	\item $\ce{4HNO3  ->[\Delta] 4NO2 ^ + O2 ^ + 2H2O}$
\end{itemize}
\paragraph{漂白性}
浓硝酸可以漂白石蕊溶液
\subsubsection{制备}
\begin{enumerate}
	\item $\ce{N2 + 3H2 <=>[{高温、高压}][{催化剂}] 2NH3}$
	\item $\ce{4NH3 + 5O2 <=>[Pt][\Delta] 4NO + 6H2O}$(催化剂一明一暗)
	\item $\ce{2NO + O2 -> 2NO2}$
	\item $\ce{3NO2 + H2O -> 2HNO3 + \underset{\text{雾}}{NO}}$
	\item ($\ce{HNO3 + NH3 -> \underset{\text{烟}}{NH4NO3}}$)
\end{enumerate}
装置:硬质石英玻璃\\
现象:催化剂一明一暗,有\textcolor[rgb]{0.827,0.286,0.184}{红棕色}气体和白色烟雾生成。
\subsubsection{固氮}
\paragraph{固氮}
将游离态的氮(氮气)转化为化合态的氮
\paragraph{自然固氮}
\subparagraph{高能固氮}
雷雨发庄稼
\begin{enumerate}
	\item $\ce{N2 + O2 ->[{放电}] 2NO}$
	\item $\ce{2NO + O2 -> 2NO2}$
	\item $\ce{3NO2 + H2O -> 2HNO3 + NO}$
\end{enumerate}
\subparagraph{生物固氮}
大豆根瘤菌
\paragraph{人工固氮}
合成氨
	
\subsection{盐}
\subsubsection{硝酸盐分解规律}
\begin{itemize}
	\item K到Mg:亚硝酸盐和氧气($\ce{2NaNO3 ->[\Delta] 2NaNO2 + O2 ^}$)
	\item Al到Cu:金属氧化物、二氧化氮和氧气($\ce{2Pb(NO3)2 ->[\Delta] 2PbO + 4NO2 ^ + O2 ^}$)
	\item Hg到Ag:金属单质、二氧化氮和氧气($\ce{2AgNO3 ->[\Delta] 2Ag + 2NO2 + O2 ^}$)
\end{itemize}
\subsubsection{铵盐分解规律}
\begin{itemize}
	\item $\ce{NH4NO3 ->[\Delta] N2O ^ + 2H2O}$
	\item $\ce{NH4HCO3 ->[\Delta] NH3 ^ + CO2 ^ + H2O}$
	\item $\ce{NH4Cl ->[\Delta] N2O ^ + HCl ^}$
	\item $\ce{(NH4)2Cr2O7 ->[\Delta] N2 ^ + CrO3 + 4H2O}$
\end{itemize}

\end{document}
