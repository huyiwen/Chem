\documentclass[a4paper]{article}
\usepackage{fontspec}
\usepackage{fontenc}
\usepackage{extarrows}
\usepackage{chemfig}
\usepackage[version=4]{mhchem}
\usepackage{mwe}
\usepackage{amsmath}
\usepackage{fancyhdr}
\usepackage{color}
\usepackage{amssymb}
\usepackage{siunitx}
\usepackage{bigfoot}
\usepackage{fancyvrb}
\usepackage{tikz}
\usepackage{expl3}
\usepackage{calc}
\usepackage{geometry}
\geometry{left=2.5cm,right=2.5cm,top=2cm,bottom=3cm}
\usetikzlibrary{graphs, positioning, quotes, shapes.geometric}
\setmainfont{Hiragino Sans GB}

\title{实验专题}
\author{胡译文}
\date{\today}

\pagestyle{fancy}
\fancyhead[L]{\footnotesize 化学笔记整理}
\fancyhead[R]{\footnotesize 胡译文}
\fancyfoot[C]{\thepage}

\makeatletter
\newcommand{\figcaption}{\def@captype{figure}\caption}
\newcommand{\tabcaption}{\def@captype{table}\caption}
\makeatother

\usepackage{hyperref}
\renewcommand\contentsname{目录}

\begin{document}
	\maketitle
	\begin{center}
		若有bug请到{\color{red}\href{https://github.com/huyiwen/Chem}{github}}上提Issue。
	\end{center}
	\renewcommand\contentsname{目录}
	\tableofcontents
	
	\clearpage
	\section{基本仪器}
	\subsection{容器}
	\subsection{量器}
	\subsubsection{容量瓶}
	\subsection{分离仪器}
	\subsection{热源}
	\subsection{其他}
	\subsubsection{温度计}
	\subsubsection{启普发生器}
	\subsection{总结}
	\subsubsection{需要验漏的仪器}
	\subsubsection{需要标注规格的仪器}
	
	\clearpage
	\section{药品}
	\subsection{保存}
	\subsection{危险标志}
	
	\clearpage
	\section{基本操作}
	\subsection{仪器的洗涤}
	\subsection{试纸的使用}
	\subsection{药品的取用}
	\subsection{配制溶液}
	\subsection{测定}
	\subsubsection{酸碱中和或氧化还原滴定}
	\subsubsection{中和反应反应热测定}
	
	\clearpage
	\section{实验}
	\subsection{Checklist}
	\subsection{装置选取}
	\subsection{实验现象}
	\subsection{收集}
	\subsection{性质探究与验证}
	\subsection{尾气处理}
	\subsection{事故处理}
	
	\clearpage
	\section{物质的检验}
	\subsection{离子检验}
	\subsubsection{焰色反应}
	\subsection{气体检验}
	\subsection{官能团检验}
	
	\clearpage
	\section{物质分离提纯}
	\subsection{物理法}
	\subsubsection{分液和萃取}
	\subsubsection{分液和萃取}
	\subsubsection{过滤}
	\subsubsection{蒸发和结晶}
	\subsubsection{蒸馏}
	\subsubsection{升华}
	\subsubsection{渗析}
	\subsection{化学法}
	\subsubsection{沉淀法}
	\paragraph{\ce{Al2O3}和\ce{MgO}}
	\begin{enumerate}
		\item \ce{NaOH}溶液:氧化铝溶解,过滤得氧化镁
		\item 稀\ce{HCl}:得氢氧化铝
		\item 加热氢氧化铝:得氧化铝
	\end{enumerate}
	\subsubsection{氧化还原法}
	\subsubsection{加热分解法}
	
\end{document}