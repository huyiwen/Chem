\documentclass[a4paper]{article}
\usepackage{fontspec}
\usepackage{fontenc}
\usepackage{extarrows}
\usepackage{chemfig}
\usepackage[version=4]{mhchem}
\usepackage{amsmath}
\usepackage{amssymb}
\usepackage{siunitx}
\usepackage{bigfoot}
\usepackage{fancyvrb}
\usepackage{expl3}
\usepackage{calc}
\usepackage{geometry}
\geometry{left=2.5cm,right=2.5cm,top=2cm,bottom=3cm}
\setmainfont{Hiragino Sans GB}

\renewcommand\contentsname{目录}

\title{元素化学笔记整理}
\author{胡译文}
\date{\today}


\begin{document}
	\maketitle
	\renewcommand\contentsname{目录}
	\tableofcontents
	\newpage
	\section{Na}
	\subsection{Na单质}
	\subsubsection{物理性质}
	\begin{itemize}
		\item 银白色固体,有金属性光泽;
		\item 密度介于水和煤油之间,用煤油或石蜡保存;
		\item 熔点低;
		\item 质地较软,可以用小刀切割。
	\end{itemize}
	\subsubsection{化学性质}
		\begin{itemize}
			\item 与非金属单质反应
			\begin{itemize}
				\item $\left\{\begin{array}{lr}
						\ce{4Na + O2 -> 2Na2O}\\
						\ce{2Na + O2 ->[\Delta] Na2O2}\\
					\end{array}\right.$
				\item $\ce{2Na + S -> Na2S}$
				\item $\ce{2Na + H2 ->[\Delta] 2NaH}$
				\item $\left\{\begin{array}{lr}
						\ce{2Na + Cl2 ->[{点燃}] 2NaCl}\\
						\ce{2Na + Br2 -> 2NaBr}\\
					\end{array}\right.$
			\end{itemize}
			\item 与水反应:\\
			$\ce{2Na + 2H2O -> 2NaOH + H2 ^}$
			\begin{itemize}
				\item 浮:钠的密度比水小
				\item 溶:反应放热,钠的熔点低
				\item 游:生成氢气推动钠
				\item 响:反应剧烈
				\item 红:生成\ce{NaOH}遇到酚酞变红
			\end{itemize}
			\item 与盐酸反应:\\
			$\ce{2Na + 2HCl -> 2NaCl + H2 ^}$
			\item 与溶液反应:\\
			钠不能与盐溶液发生置换反应,其反应的实质是先与水反应,产物再和盐反应.
			\begin{itemize}
				\item 钠与硫酸铜溶液
				$\left\{\begin{array}{lr}
					\ce{2Na + 2H20 -> 2NaOH + H2 ^}\\
					\ce{2NaOH + CuSO4 -> Na2SO4 + Cu(OH)2 v}\\
				\end{array}\right.$
			\end{itemize}
			\item 与\ce{CO2}反应:\\
			$\left\{\begin{array}{lr}
				\ce{4Na + CO2 ->[\Delta] 2Na2O + C}\\
				\ce{4Na + 3CO2 ->[\Delta] 2Na2CO3 + C}\\
			\end{array}\right.$
		\end{itemize}
	\subsubsection{钠的制取}
	$\left\{\begin{array}{lr}
		\ce{2NaCl(l) ->[{电解}] 2Na + Cl2 ^}\\
		\ce{2NaOH(l) ->[{电解}] 2Na + O2 ^ + H2 ^}\\
	\end{array}\right.$
	\subsubsection{钠的用途}
	\begin{itemize}
		\item 冶炼金属:$\ce{4Na + TiCl4(l) -> 4NaCl + Ti}$
		\item 原子反应导热剂
		\item 钠光灯
	\end{itemize}
\end{document}
